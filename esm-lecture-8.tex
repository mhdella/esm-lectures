% NB: use pdflatex to compile NOT pdftex.  Also make sure youngtab is
% there...

% converting eps graphics to pdf with ps2pdf generates way too much
% whitespace in the resulting pdf, so crop with pdfcrop
% cf. http://www.cora.nwra.com/~stockwel/rgspages/pdftips/pdftips.shtml




\documentclass[10pt,aspectratio=169,dvipsnames]{beamer}

\usetheme[color/block=transparent]{metropolis}

\usepackage[absolute,overlay]{textpos}
\usepackage{booktabs}
\usepackage[utf8]{inputenc}

\usepackage{tikz}


\usepackage[scale=2]{ccicons}

\usepackage[official]{eurosym}


%use this to add space between rows
\newcommand{\ra}[1]{\renewcommand{\arraystretch}{#1}}


\newcommand{\R}{\mathbb{R}}



\setbeamerfont{alerted text}{series=\bfseries}
\setbeamercolor{alerted text}{fg=Mahogany}
\setbeamercolor{background canvas}{bg=white}



\def\l{\lambda}
\def\m{\mu}
\def\d{\partial}
\def\cL{\mathcal{L}}
\def\co2{CO${}_2$}
\def\bra#1{\left\langle #1\right|}
\def\ket#1{\left| #1\right\rangle}
\newcommand{\braket}[2]{\langle #1 | #2 \rangle}
\newcommand{\norm}[1]{\left\| #1 \right\|}
\def\corr#1{\Big\langle #1 \Big\rangle}
\def\corrs#1{\langle #1 \rangle}


\def\mw{\text{ MW}}
\def\mwh{\text{ MWh}}
\def\emwh{\text{ \euro/MWh}}

\newcommand{\ubar}[1]{\text{\b{$#1$}}}


% for sources http://tex.stackexchange.com/questions/48473/best-way-to-give-sources-of-images-used-in-a-beamer-presentation

\setbeamercolor{framesource}{fg=gray}
\setbeamerfont{framesource}{size=\tiny}



\newcommand{\source}[1]{\begin{textblock*}{5cm}(10.5cm,8.35cm)
    \begin{beamercolorbox}[ht=0.5cm,right]{framesource}
        \usebeamerfont{framesource}\usebeamercolor[fg]{framesource} Source: {#1}
    \end{beamercolorbox}
\end{textblock*}}

\usepackage{hyperref}

\usepackage{tikz}


\usepackage[europeanresistors,americaninductors]{circuitikz}


%\usepackage[pdftex]{graphicx}


\graphicspath{{graphics/}}

\DeclareGraphicsExtensions{.pdf,.jpeg,.png,.jpg,.gif}



\def\goat#1{{\scriptsize\color{green}{[#1]}}}



\let\olditem\item
\renewcommand{\item}{%
\olditem\vspace{5pt}}


\title{Energy System Modelling\\ Summer Semester 2020, Lecture 8}
%\subtitle{---}
\author{
  {\bf Dr. Tom Brown}, \href{mailto:tom.brown@kit.edu}{tom.brown@kit.edu}, \url{https://nworbmot.org/}\\
  \emph{Karlsruhe Institute of Technology (KIT), Institute for Automation and Applied Informatics (IAI)}
}

\date{}


\titlegraphic{
  \vspace{0cm}
  \hspace{10cm}
    \includegraphics[trim=0 0cm 0 0cm,height=1.8cm,clip=true]{kit.png}

\vspace{5.1cm}

  {\footnotesize

  Unless otherwise stated, graphics and text are Copyright \copyright Tom Brown, 2020.
  Graphics and text for which no other attribution are given are licensed under a
  \href{https://creativecommons.org/licenses/by/4.0/}{Creative Commons
  Attribution 4.0 International Licence}. \ccby}
}

\begin{document}

\maketitle


\begin{frame}

  \frametitle{Table of Contents}
  \setbeamertemplate{section in toc}[sections numbered]
  \tableofcontents[hideallsubsections]
\end{frame}


\section{Optimisation Revision}

\begin{frame}
  \frametitle{Optimisation problem}


We have an \alert{objective function} $f: \R^k \to \R$
\begin{equation*}
  \max_{x} f(x)
\end{equation*}
[$x = (x_1, \dots x_k)$] subject to some \alert{constraints} within $\R^k$:
\begin{align*}
  g_i(x) & = c_i \hspace{1cm}\leftrightarrow\hspace{1cm} \l_i \hspace{1cm} i = 1,\dots n \\
  h_j(x) & \leq d_j \hspace{1cm}\leftrightarrow\hspace{1cm} \m_j \hspace{1cm} j = 1,\dots m
\end{align*}

$\l_i$ and $\m_j$ are the \alert{KKT multipliers} we introduce for
each constraint equation; they measure the change in the objective value of the optimal solution obtained by relaxing the constraints (for this reason they are also called \alert{shadow prices}).

\end{frame}



\begin{frame}
  \frametitle{KKT conditions}

The \alert{Karush-Kuhn-Tucker (KKT) conditions} are necessary conditions that an optimal solution $x^*,\m^*,\l^*$ always satisfies (up to some regularity conditions):
\begin{enumerate}
\item \alert{Stationarity}: For $l = 1,\dots k$
  \begin{equation*}
  \frac{\d \cL}{\d x_l} =   \frac{\d f}{\d x_l} - \sum_i \l_i^* \frac{\d g_i}{\d x_l}  - \sum_j \m_j^* \frac{\d h_j}{\d x_l} = 0
  \end{equation*}
    \item \alert{Primal feasibility}:
      \begin{align*}
        g_i(x^*) & = c_i \\
        h_j(x^*) &\leq d_j
      \end{align*}
    \item \alert{Dual feasibility}: $\m_j^* \geq 0$
    \item \alert{Complementary slackness}: $\m_j^* (h_j(x^*) - d_j) = 0$
\end{enumerate}

\end{frame}

\begin{frame}
  \frametitle{min/max and signs}

  If the problem is a \alert{maximisation} problem (e.g. welfare maximisation), then
  \alert{$\m_j^* \geq 0$} since $\m_j = \frac{\d \cL}{\d d_j}$ and if we
  increase $d_j$ in the constraint $h_j(x) \leq d_j$, then the
  feasible space can only get bigger. Since if $X \subseteq X'$
  \begin{equation*}
 \max_{x\in X} f(x ) \leq     \max_{x\in X'} f(x)
  \end{equation*}
  then the objective value at the optimum point can only get bigger, and thus $\m_j^* \geq 0$. (If $d_j \to \infty$
  then the constraint is no longer binding, if $d_j \to -\infty$ then
  the feasible space vanishes.)

  If however the problem is a \alert{minimisation} problem (e.g. cost minimisation) then we can use
  \begin{equation*}
      \min_{x\in X} f(x) = - \max_{x\in X} \left[ -f(x)\right]
  \end{equation*}
  We can keep our definition of the Lagrangian and almost all the KKT
  conditions, but we have a change of sign \alert{$\m_j^* \leq 0$}, since
  \begin{equation*}
   \min_{x\in X} f(x ) \geq     \min_{x\in X'} f(x)
  \end{equation*}
  The $\l_i^*$ also change sign.

\end{frame}




\section{Welfare maximisation revision}



\begin{frame}{KKT and Welfare Maximisation 1/2}

  Apply KKT now to maximisation of total economic welfare:
  \begin{align*}
    \max_{\{d_b\}, \{g_s\}} f(\{d_b\}, \{g_s\}) = \left[ \sum_b U_b (d_b)  -  \sum_s C_s (g_s) \right]
  \end{align*}
  subject to the balance constraint:
  \begin{align*}
    g(\{d_b\}, \{g_s\}) = \sum_b d_b -  \sum_s g_s  = 0 \hspace{1cm} \leftrightarrow \hspace{1cm} \l
  \end{align*}
  and any other constraints (e.g. limits on generator capacity, etc.).

  Our optimisation variables are $\{x\} = \{d_b\} \cup \{g_s\}$.

  We get from KKT stationarity at the optimal point:
  \begin{align*}
    0 & =   \frac{\d f}{\d d_b} - \sum_b \l^* \frac{\d g}{\d d_b} = U_b'(d_b^*) - \l^* = 0 \\
    0 & =   \frac{\d f}{\d g_s} - \sum_s \l^* \frac{\d g}{\d g_s} = -C_s'(g_s^*) + \l^* = 0
  \end{align*}


\end{frame}





\begin{frame}{KKT and Welfare Maximisation 2/2}

  So at the optimal point of maximal total economic welfare we get the
  same result as if everyone maximises their own welfare separately based on the price $\l^*$:
  \begin{align*}
 U_b'(d_b^*) =  \l^*  \\
 C_s'(g_s^*) = \l^*
  \end{align*}

  This is the CENTRAL result of microeconomics.

  If we have further inequality constraints that are binding (e.g. capacity constraints), then
  these equations will receive additions with $\m_i^* > 0$.


\end{frame}




\section{Optimise Single Node with Linear Generation Costs and Demand Utility}


\begin{frame}[fragile]
  \frametitle{Simplified world: linear generation costs, linear demand utility}

  We will now turn to a simpler world: all the generator cost
  functions are linear
  \begin{align*}
    C_s(g_s) = o_s g_s
  \end{align*}
  and each generator has limited output $0 \leq g_s \leq G_s$. The marginal cost function is a constant $C_s'(g_s) = o_s$.


  The quantity $G_s$ and marginal cost $o_s$ define a \alert{supply offer}.


  All the consumer utility functions are also linear
  \begin{align*}
    U_b(d_b)  = v_b d_b
  \end{align*}
  and each consumer has limited consumption $0 \leq d_b \leq D_b$. The marginal utility function is a constant $U_b'(d_b)  = v_b$.


  The quantity $D_b$ and marginal utility $v_b$ define a \alert{demand bid}.


\end{frame}



\begin{frame}{Supply-demand linear example: generator offers}

  Example from Kirschen and Strbac pages 56-58.

  The following generators offer into the market for the hour between 0900 and 1000 on 20th April 2016:
  \ra{1.1}
  \begin{table}[!t]
    \begin{tabular}{lrr}
      \toprule
      Company & Quantity [MW] &  Marginal cost [\$/MWh]\\
      \midrule
      Red & 200 & 12 \\
      Red & 50 & 15 \\
      Red & 150 & 20 \\
      Green & 150 & 16 \\
      Green & 50 & 17 \\
      Blue & 100 & 13 \\
      Blue & 50 & 18 \\
      \bottomrule
    \end{tabular}
  \end{table}


\end{frame}



\begin{frame}{Supply-demand linear example: consumer bids}

  The following consumers make bids for the same period:
  \ra{1.1}
  \begin{table}[!t]
    \begin{tabular}{lrr}
      \toprule
      Company & Quantity [MW] &  Marginal utility [\$/MWh]\\
      \midrule
      Yellow & 50 & 13 \\
      Yellow & 100 & 23 \\
      Purple & 50 & 11 \\
      Purple & 150 & 22 \\
      Orange & 50 & 10 \\
      Orange & 200 & 25 \\
      \bottomrule
    \end{tabular}
  \end{table}


\end{frame}




\begin{frame}{Supply-demand example: Curve}

  If the bids and offers are stacked up in order, the supply and
  demand curves meet with a demand of 450~MW at a system marginal
  price of $\l^* = 16$~\$/MWh.

  \centering


  \includegraphics[width=10cm]{supply-bid-example}


  \source{Kirschen \& Strbac}



\end{frame}





\begin{frame}{Supply-demand example: Revenue and Expenses}

  Dispatch and revenue/expense of each company:
  \ra{1.1}
  \begin{table}[!t]
    \begin{tabular}{l|rr|rr}
      \toprule
      Company & Production& Consumption & Revenue& Expense\\
       & [MWh] & [MWh] & [\$] & [\$] \\
      \midrule
      Red & 250 &  & 4000 & \\
      Blue & 100 & & 1600 & \\
      Green & 100 & & 1600 & \\
      Orange && 200 && 3200 \\
      Yellow && 100 && 1600 \\
      Purple && 150 && 2400 \\
      \midrule
      Total & 450 & 450 & 7200 & 7200 \\
      \bottomrule
    \end{tabular}
  \end{table}


\end{frame}





\begin{frame}[fragile]
  \frametitle{Simplified world: linear generation costs, single inelastic demand}

  For the analysis of the KKT equations, we will simplify even further.

  We consider a single demand bid of volume $D$ so that the demand does not respond to price changes (i.e. the demand is \alert{inelastic}) up to a very high marginal utility $v >> o_s\,\, \forall s$, i.e.
  \begin{align*}
    U(d) = vd
  \end{align*}
  for $d\leq D$.

  $v$ is sometimes called the \alert{Value Of Lost Load (VOLL)}.
\end{frame}


\begin{frame}[fragile]
  \frametitle{Simplify representation of consumers and generators}

  In this case we get for our welfare maximisation:
  \begin{align*}
    \max_{d, \{g_s\} }\left[ vd  -  \sum_s o_s g_s \right]
  \end{align*}
  subject to:
  \begin{align*}
    d -  \sum_s g_s   = 0 \hspace{1cm} & \leftrightarrow \hspace{1cm} \l \\
    d  \leq D  \hspace{1cm}  & \leftrightarrow \hspace{1cm} \m \\
        g_s   \leq  G_s  \hspace{1cm}& \leftrightarrow\hspace{1cm} \bar{\m}_s \\
    - g_s   \leq  0   \hspace{1cm}& \leftrightarrow\hspace{1cm} \ubar{\m}_s
  \end{align*}

\end{frame}



\begin{frame}{Simplest example: one generator type, inelastic demand}

  Suppose all generators have the same marginal cost $o$ and we
  represent their total dispatch by $g$ and total capacity by $G$
  \begin{align*}
    \max_{d, g }\left[ vd  -  og\right]
  \end{align*}
  such that:
  \begin{align*}
    d - g & = 0  \hspace{1cm}\leftrightarrow\hspace{1cm} \l \\
    d & \leq D  \hspace{1cm} \leftrightarrow \hspace{1cm} \m \\
    g  & \leq  G  \hspace{1cm}\leftrightarrow\hspace{1cm} \bar{\m} \\
    - g  & \leq  0  \hspace{1cm}\leftrightarrow\hspace{1cm} \ubar{\m}
  \end{align*}

\end{frame}


\begin{frame}{Simplest example: one generator type, inelastic demand}

  If $D < G$ then since $v >> o$, it will be always be welfare-maximising to dispatch to satisfy the load, i.e.
  \begin{equation*}
    g^* = d^* = D
  \end{equation*}
  If the demand is non-zero then since $g^* > 0$ by complementarity we
  have $\ubar{\m}^* = 0$. Since $D < G$ then $g^* < G$ and
  by complementarity we have $\bar{\m}^* = 0$. To compute $\l^*$ we use stationarity:
  \begin{align*}
    0 & = \frac{\d \cL}{\d g } =   \frac{\d f}{\d g} - \sum_i \l_i^* \frac{\d g_i}{\d g}  - \sum_j \m_j^* \frac{\d h_j}{\d g} = - o + \l^* - \bar{\m}^* + \ubar{\m}^*
  \end{align*}
  Thus $\l^* = o$, which is the cost per unit of supplying extra demand. The \alert{generator sets the price}. There is no generator profit and a large consumer surplus.

  For the load $\m^*$ can be non-zero because $d^*=D$:
    \begin{align*}
      0 & = \frac{\d \cL}{\d d} =   \frac{\d f}{\d d} - \sum_i \l_i^* \frac{\d g_i}{\d d}  - \sum_j \m_j^* \frac{\d h_j}{\d d} = v - \l^* - \m^*
    \end{align*}
    $\m^* = v - \l^*$ is the marginal benefit of each increase in demand.
\end{frame}



\begin{frame}{Simplest example: one generator type, inelastic demand}

  For the case $D < G$:

  \includegraphics[width=12cm]{supply-bigger-demand-single.pdf}

\end{frame}

\begin{frame}{Simplest example: one generator type, inelastic demand}

  If $D > G$ then the generator will dispatch up to its maximum capacity
  \begin{equation*}
    g^* = d^* = G
  \end{equation*}
  For its lower limit we have $\ubar{\m}^* = 0$. From stationarity:
  \begin{align*}
    0 & = \frac{\d \cL}{\d g } =   \frac{\d f}{\d g} - \sum_i \l_i^* \frac{\d g_i}{\d g}  - \sum_j \m_j^* \frac{\d h_j}{\d g} = - o + \l^* - \bar{\m}^* + \ubar{\m}^*
  \end{align*}
  Thus $\l^* = o + \bar{\m}^*$. To find $\l^*$ we have to look at the demand:
    \begin{align*}
      0 & = \frac{\d \cL}{\d d} =   \frac{\d f}{\d d} - \sum_i \l_i^* \frac{\d g_i}{\d d}  - \sum_j \m_j^* \frac{\d h_j}{\d d} = v - \l^* - \m^*
    \end{align*}
  Since $d^* < D$, $\m^* = 0$, $\l^* = v$ and thus $\bar{\m}^* = v-o$, which is the marginal benefit of increasing the generator capacity $G$. The \alert{demand sets the price}. There is no consumer surplus and the generator makes a large profit.

\end{frame}


\begin{frame}{Simplest example: one generator type, inelastic demand}

  For the case $D > G$:

  \includegraphics[width=12cm]{demand-bigger-supply.pdf}

\end{frame}


\begin{frame}{Next simplest example: several generators, fixed demand}

  Suppose we have several generators with dispatch $g_s$ and strictly ordered
  operating costs $o_s$ such that $o_s < o_{s+1}$. We now maximise
  \begin{align*}
    \max_{\{d, g_s\}}  \left[ vd  -  \sum_s o_s g_s \right]
  \end{align*}
  such that
  \begin{align*}
    d -  \sum_s g_s  & = 0 \hspace{1cm} \leftrightarrow \hspace{1cm} \l \\
    d & \leq D  \hspace{1cm} \leftrightarrow \hspace{1cm} \m \\
        g_s  & \leq  G_s  \hspace{1cm}\leftrightarrow\hspace{1cm} \bar{\m}_s \\
    - g_s  & \leq  0  \hspace{1cm}\leftrightarrow\hspace{1cm} \ubar{\m}_s
  \end{align*}


\end{frame}

\begin{frame}{Next simplest example: several generators, fixed demand}

  Stationarity gives us for each generator $g_s$:
  \begin{align*}
    0  = \frac{\d \cL}{\d g_s} = -o_s + \l^* - \bar{\m}^*_s + \ubar{\m}^*_s
  \end{align*}
  and from complementarity we get
  \begin{align*}
    \bar{\m}_s(g_s^* - G_s)  = 0 \hspace{2cm}    \ubar{\m}_sg_s^*  = 0
  \end{align*}
  We can see by inspection that we will dispatch the cheapest
  generation first. Suppose that we have enough generation for the
  demand, i.e. $D < \sum_s G_s$. [If $D > \sum_s G_s$ we have the same
    situation as for a single generator, i.e. $\l^* = v$, so that the demand sets the price.]

  Find the generator $m$ on the margin where the supply curve
  intersects with the demand $D$, i.e. the $m$ where $\sum_{s=1}^{m-1}
  G_s < D < \sum_{s=1}^{m} G_s$.

  For $s \leq m-1$ we have $g_s^* = G_s$, $\ubar{\m}^*_s = 0$,
  $\bar{\m}^*_s =  \l^* - o_s$.

  For $s = m$ we have $g_m^* = D - \sum_{s=1}^{m-1} G_s$ to cover
  what's left of the demand. Since $0 < g_m^* < G_m$ we have
  $\ubar{\m}^*_m = \bar{\m}^*_m = 0$ and thus $\l^* = o_m$.


\end{frame}



\begin{frame}{Next simplest example: several generators, fixed demand}

  Specific example of two generators with $G_1 = 300$~MW, $G_2 =
  400$~MW, $o_1 = 10$~\euro/MWh, $o_2 = 30$~\euro/MWh and $D = 500$~MW.

  In this case $m=2$, $g_1^* = G_1 = 300$~MW, $g_2^* = d - G_1 = 200$~MW,
  $\l^* = o_2$, $\ubar{\m}_i = 0$, $\bar{\m}_2 = 0$ and $\bar{\m}_1 =
  o_2 - o_1$.

  \centering
  \includegraphics[width=8cm]{supply-demand-two.pdf}




\end{frame}





\begin{frame}{From welfare maximisation to cost minimisation}

  For the case $D > \sum_s G_s$ we can instead imagine that the demand
  is rigidly fixed to $D$ and that instead we have a dummy generator
  with dispatch $g_d = D-\sum_s G_s$ that represents \alert{load shedding}.
  In this case we can substitute $d = D - g_d$ to get
  \begin{align*}
    \max_{\{g_d, g_s\}}  \left[ vD - vg_d  -  \sum_s o_s g_s \right]
  \end{align*}
  such that
  \begin{align*}
    D - g_d -  \sum_s g_s  & = 0 \hspace{1cm} \leftrightarrow \hspace{1cm} \l \\
        g_s  & \leq  G_s  \hspace{1cm}\leftrightarrow\hspace{1cm} \bar{\m}_s \\
    - g_s  & \leq  0  \hspace{1cm}\leftrightarrow\hspace{1cm} \ubar{\m}_s
  \end{align*}

  Since $vD$ is a constant, we can use
      $\max_{x\in X} \left[ -f(x)\right] = - \min_{x\in X} f(x)$
  to recast this as a minimisation of the total generator costs,
  absorbing $g_d$ into the set $\{g_s\}$. The constant $vD$ is dropped.

\end{frame}


\begin{frame}{From welfare maximisation to cost minimisation}

 We have  turned the maximisation of total welfare into \alert{cost minimisation}:
  \begin{align*}
    \min_{\{g_s\}}  \sum_s o_s g_s
  \end{align*}
  such that:
  \begin{align*}
    \sum_s g_s - d & = 0  \hspace{1cm}\leftrightarrow\hspace{1cm} \l \\
        g_s  & \leq  G_s  \hspace{1cm}\leftrightarrow\hspace{1cm} \bar{\m}_s \\
    - g_s  & \leq  0  \hspace{1cm}\leftrightarrow\hspace{1cm} \ubar{\m}_s
  \end{align*}

  The most expensive generator has $o_s = v$ and $G_s = \infty$ and
  represents \alert{load shedding}.

  We've replaced the symbol $D$  with $d$ for simplicity going forward ($d$ is now a constant).



  NB: Because the signs of the KKT multipliers change when we go from
  maximisation to minimisation, we've also changed the sign of the
  balance constraint to keep the marginal price $\l$ positive.
\end{frame}


\section{Optimise nodes in a network}



\begin{frame}[fragile]
  \frametitle{Welfare optimisation for several nodes in a network}

  Now let's suppose we have several nodes $i$ with different loads and
  different generators, with flows $f_\ell$ in the network lines $\ell$.

  Now we have additional optimisation variables $f_\ell$ AND
  additional constraints for welfare maximisation:
  \begin{align*}
    \max_{\{d_{i,b}\},\{g_{i,s}\},\{f_\ell\}}\left[\sum_{i,b} U_{i,b}(d_{i,b}) - \sum_{i,s} C_{i,s} (g_{i,s}) \right]
  \end{align*}
  such that demand is met either by generation or by the network at each node $i$
  \begin{align*}
    \sum_{b} d_{i,b} - \sum_{s} g_{i,s} +  \sum_\ell K_{i\ell}f_\ell  = 0 \hspace{1cm}\leftrightarrow\hspace{1cm} \l_i
  \end{align*}
  Note there is now a \alert{market price for each node}. As before, generator constraints are satisified
    \begin{align*}
        g_{i,s}  & \leq  G_{i,s}  \hspace{1cm}\leftrightarrow\hspace{1cm} \bar{\m}_{i,s} \\
    - g_{i,s}  & \leq  0  \hspace{1cm}\leftrightarrow\hspace{1cm} \ubar{\m}_{i,s}
  \end{align*}



\end{frame}


\begin{frame}[fragile]
  \frametitle{Linear cost minimisation at several nodes in a network}

  For cost minimisation we have a fixed load $d_i$ at each node, and
  absorb load-shedding above a value $v$ into a dummy generator.

  Now we minimise over $f_\ell$ and $g_{i,s}$ for the case of linear cost functions:
  \begin{align*}
    \min_{\{g_{i,s}\},\{f_\ell\}}  \sum_{i,s} o_{i,s} g_{i,s}
  \end{align*}
  such that demand is met either by generation or by the network at each node $i$
  \begin{align*}
    \sum_{s} g_{i,s} - d_i = \sum_\ell K_{i\ell}f_\ell  \hspace{1cm}\leftrightarrow\hspace{1cm} \l_i
  \end{align*}
  and generator constraints are satisified
    \begin{align*}
        g_{i,s}  & \leq  G_{i,s}  \hspace{1cm}\leftrightarrow\hspace{1cm} \bar{\m}_{i,s} \\
    - g_{i,s}  & \leq  0  \hspace{1cm}\leftrightarrow\hspace{1cm} \ubar{\m}_{i,s}
  \end{align*}



\end{frame}

\begin{frame}[fragile]
  \frametitle{Several generators at different nodes in a network}

  In addition we have constraints on the line flows.

  First, they have to satisfy Kirchoff's Voltage Law around each closed cycle $c$:
  \begin{align*}
    \sum_{c} C_{\ell c} x_\ell f_\ell = 0  \hspace{1cm}\leftrightarrow\hspace{1cm} \l_c
  \end{align*}
  and in addition the flows cannot overload the thermal limits, $|f_\ell| \leq F_\ell$
  \begin{align*}
    f_\ell \leq F_\ell  \hspace{1cm}\leftrightarrow\hspace{1cm} \bar{\m}_\ell \\
        - f_\ell \leq F_\ell  \hspace{1cm}\leftrightarrow\hspace{1cm} \ubar{\m}_\ell
  \end{align*}

\end{frame}


\begin{frame}[fragile]
  \frametitle{Simplest example: two nodes connected by a single line}

  At node 1 we have demand of $d_1 = 100$~MW and a generator with
  costs $o_1 = 10$~\euro/MWh and a capacity of $G_1 = 300$~MW.

  At node 2 we have demand of $d_2 = 100$~MW and a generator with
  costs $o_2 = 20$~\euro/MWh and a capacity of $G_2 = 300$~MW.

  What happens if the capacity of the line connecting them is $F = 0$?

  What about $F = 50$~MW?

  What about $F = \infty$?

\end{frame}


\begin{frame}[fragile]
  \frametitle{Simplest example: two nodes connected by a single line}

\centering
\begin{circuitikz}
  \draw [ultra thick] (1,13) node[anchor=south]{1} -- (4,13);
  \draw(2.5,13) |- +(0,0.5) to [short,i^=$f$] +(5,0.5) |- +(0,-0.5);
  \draw [ultra thick] (6,13) node[anchor=south]{2} -- (9,13);
  \draw (1.5,13) -- +(0,-0.5) node[sground]{};
  \draw (3,12) node[vsourcesinshape, rotate=270](V2){}
  (V2.left) -- +(0,0.6);
  \draw (1.5,11) node{$d_1$};
  \draw (3,11) node{$g_{1}$};
  \draw (6.5,13) -- +(0,-0.5) node[sground]{};
  \draw (8,12) node[vsourcesinshape, rotate=270](V2){}
  (V2.left) -- +(0,0.6);
  \draw (6.5,11) node{$d_2$};
  \draw (8,11) node{$g_{2}$};

\end{circuitikz}


\centering
\begin{columns}[T]
  \begin{column}{5cm}

    \centering

 $g_1 - d_1 = f \quad \leftrightarrow \quad \l_1$

    \hspace{1cm}

    $d_1 =$ 100~MW

    $G_1 =$ 300~MW

    $o_1 = 10$~\euro/MWh

  \end{column}
  \begin{column}{5cm}

    \centering
    $g_2 - d_2 =  -f   \quad \leftrightarrow \quad \l_2$

    \hspace{1cm}

    $d_2 =$ 100~MW

    $G_2 =$ 300~MW

    $o_2 = 20$~\euro/MWh

  \end{column}
\end{columns}
\end{frame}

\begin{frame}[fragile]
  \frametitle{Simplest example: two nodes connected by a single line}

  Out optimisation problem has objective function:
  \begin{equation*}
    \min_{g_1,g_2,f}  \left[o_1 g_1 + o_2 g_2 \right]
  \end{equation*}
  subject to the following constraints:
  \begin{align*}
    g_1 - d_1  = f \hspace{1cm}\leftrightarrow\hspace{1cm} \l_1 \\
    g_2 - d_2  = -f \hspace{1cm}\leftrightarrow\hspace{1cm} \l_2 \\
  g_1 \leq G_1  \hspace{1cm}\leftrightarrow\hspace{1cm} \bar{\m}_1 \\
    - g_1 \leq 0  \hspace{1cm}\leftrightarrow\hspace{1cm} \ubar{\m}_1  \\
  g_2 \leq G_2  \hspace{1cm}\leftrightarrow\hspace{1cm} \bar{\m}_2  \\
  - g_2 \leq 0  \hspace{1cm}\leftrightarrow\hspace{1cm} \ubar{\m}_2  \\
    f \leq F  \hspace{1cm}\leftrightarrow\hspace{1cm} \bar{\m}  \\
    - f \leq F  \hspace{1cm}\leftrightarrow\hspace{1cm} \ubar{\m}
\end{align*}
\end{frame}

\begin{frame}[fragile]
  \frametitle{Two nodes: Case $F = 0$ }

  For the case $F = 0$ the nodes are like two separated islands, $f^* = 0$.

  The generator on each island provides the demand separately, so:
  \begin{equation*}
    g_1^* = d_1  \hspace{2cm} \textrm{and} \hspace{2cm} g_2^* = d_2
  \end{equation*}

  Neither generator has any binding constraints, since in each case the demand (100~MW) is less than the generator capacity (300~MW), so
  \begin{equation*}
    \bar{\m}_1^* = \ubar{\m}_1^* =     \bar{\m}_2^* = \ubar{\m}_2^* = 0
  \end{equation*}

  From stationarity for each site we get
  \begin{equation*}
    0 = \frac{\d \cL}{\d g_{i}} = o_i - \l_i^*  -  \bar{\m}_i^* + \ubar{\m}_i^*
  \end{equation*}

  Thus we have at each site $\l_i^* = o_i$, as if we had optimised the nodes separately.


\end{frame}


\begin{frame}[fragile]
  \frametitle{Two nodes: Case $F =$ 50~MW }

  For the case $F = $50~MW the cheaper node 1 will export to the more
  expensive node 2 as much as the restricted capacity $F$ allows:
  \begin{equation*}
    f^* = F = 50\textrm{ MW}
  \end{equation*}
  Generator 1 covers 50~MW of the demand from node 2:
  \begin{equation*}
    g_1^* = d_1+f^* = 150\textrm{ MW}  \hspace{2cm} \textrm{and} \hspace{2cm} g_2^* = d_2 - f^* = 50\textrm{ MW}
  \end{equation*}
  Neither generator has any binding constraints, so
  \begin{equation*}
    \bar{\m}_1^* = \ubar{\m}_1^* =     \bar{\m}_2^* = \ubar{\m}_2^* = 0
  \end{equation*}
  and thus we have again different prices at each $\l_i^* = o_i$. For the flow:
  \begin{equation*}
    0 = \frac{\d \cL}{\d f} = 0 +  \l_1^* - \l_2^*  -  \bar{\m}^* + \ubar{\m}^*
  \end{equation*}
  Only the upper limit is binding, so we get $\ubar{\m}^* = 0$ and
    $\bar{\m}^* = \l_1^* - \l_2^* = o_1 - o_2 = -10 $ \euro/MWh.

$\bar{\m}^*$ is the cost reduction if we expand the transmission capacity $F$ by $\varepsilon$, allowing us to substitute some of the expensive generation at node 2 with cheap generation from node 1.


\end{frame}


\begin{frame}[fragile]
  \frametitle{Two nodes: Case $F = \infty$ }

  For the case $F = \infty$ we have unrestricted capacity, so it is
  like merging the two nodes to one node. Now all the demand is
  covered by the cheapest node:
  \begin{equation*}
    f^* = d_2 = 100\textrm{ MW}
  \end{equation*}
  Generator 1 covers all the demand:
  \begin{equation*}
    g_1^* = d_1+d_2  = 200\textrm{ MW} \hspace{2cm} \textrm{and} \hspace{2cm} g_2^* = 0
  \end{equation*}
  Only generator 2 has a non-zero KKT multiplier, so at node 1 we have
  $\l_1^* =o_1$ and at node 2 we have:
  \begin{equation*}
   \ubar{\m}_2^* = \l_2^* - o_2
  \end{equation*}
  From KKT for the flow $f$ we have no constraints so $\bar{\m}^* = \ubar{\m}^* = 0$ and from stationarity
  \begin{equation*}
    0 = \frac{\d \cL}{\d f} = 0 +  \l_1^* - \l_2^*  -  \bar{\m}^* + \ubar{\m}^*
  \end{equation*}
  i.e. $\l_1^* = \l_2^*$. We have price equalisation, as if it were a single node.

\end{frame}



\begin{frame}{Two node: demand pagements versus generation revenue}

  Now let's compare for our examples what each demand pays $\l_i^*d_i$ and what each generator receives as revenue $\l_i^*g_i^*$ from each market.
  \ra{1.1}
  \begin{table}[!t]
    \begin{tabular}{rrrrrrrrr}
      \toprule
      Case & $\l_1^*$  & $\l_2^*$ & $\l_1^*d_1$ & $\l_2^*d_2$ & $\sum_i \l_i^*d_i$ & $\l_1^*g_1^*$  & $\l_2^*g_2^*$ & $\sum_i \l_i^*g_i^*$ \\
       & [\euro/MWh] &  [\euro/MWh] & [\euro/h] & [\euro/h] & [\euro/h] & [\euro/h] & [\euro/h] & [\euro/h] \\
      \midrule
      $F = 0$ & 10 & 20 & 1000 & 2000 & 3000 & 1000 & 2000 & 3000 \\
      $F = 50$ & 10 & 20 & 1000 & 2000 & 3000 & 1500 & 1000 & 2500 \\
      $F = \infty$ & 10 & 10 & 1000 & 1000 & 2000 & 2000 & 0 & 2000 \\
%      $F = 50$ & 3000 & 2500 & 10 & 50 & 500
%      $F = \infty$ & 2000 & 2000 & 0 & 100 & 0  \\
      \bottomrule
    \end{tabular}
  \end{table}

  NB: In the case with $F=50$, total demand payments are 3000~\euro/h, whereas the generators are only receiving 2500~\euro/h.

  Where is the missing money (500~\euro/h) going?

  Answer: to the network operator for service of doing arbitrage, buying low and selling high.

\end{frame}




\begin{frame}{Congestion rent}

Due to the congestion of the transmission line, the marginal cost of producing electricity can be different at node 1 and node 2. The competitive price at node 2 is higher than at node 1 -- this corresponds to \alert{locational marginal pricing}, or \alert{nodal pricing}.

Since consumers pay and generators get paid the price in their local market, in case of congestion there is a difference between the total payment of consumers and the total revenue of producers -- this is the \alert{merchandising surplus} or \alert{congestion rent}, collected by the network operator. For each line it is given by the price difference in both regions times the amount of power flow between them:
\begin{align*}
\text{Congestion rent =}\:\Delta \lambda\times f
\end{align*}


\end{frame}


\begin{frame}{Congestion rent: Two node example}

  Returning to our two node example:
  \ra{1.1}
  \begin{table}[!t]
    \begin{tabular}{rrrrrr}
      \toprule
      Case & Demand pays & Generator gets & $\l_2^* - \l_1^*$ & flow $f$ & Cong. rent \\
       & [\euro/h] & [\euro/h] & [\euro/MWh] & [MW] & [\euro/h] \\
      \midrule
      $F = 0$ & 3000 & 3000 & 10 & 0 & 0 \\
      $F = 50$ & 3000 & 2500 & 10 & 50 & 500 \\
      $F = \infty$ & 2000 & 2000 & 0 & 100 & 0 \\
      \bottomrule
    \end{tabular}
  \end{table}

To get a congestion rent, we need congestion to cause a price
difference between the nodes, as well as a non-zero flow between the
nodes.

\end{frame}



\begin{frame}[fragile]
  \frametitle{Congestion rent}

  In this example we saw that the sum of what consumers pay does not always equal the sum of generator revenue.

  In fact if we take the balance constraint and sum it weighted by the market price at each node we find
  \begin{align*}
    \sum_i \l_i^* d_i -   \sum_i \l_i^* \sum_{s} g^*_{i,s} = -\sum_i \l_i^* \sum_\ell K_{i\ell}f^*_\ell
  \end{align*}

  The quantity for each $\ell$
  \begin{equation*}
    -f_\ell^*\sum_i K_{i\ell} \l_i^* = f_\ell (\l_{\textrm{end}}^* - \l_{\textrm{start}}^*)
  \end{equation*}
  is called the \alert{congestion rent} and is the money the network
  operator receives for transferring power from a low price node (start)
  to a high price node (end), `buy it low, sell it high'.

  It is zero if: a) the flow is zero or b) the price difference is zero.

\end{frame}





\section{The European Market}


\begin{frame}
  \frametitle{Existing bidding zones}



\begin{columns}[T]
  \begin{column}{5cm}

    \includegraphics[width=5.8cm]{european_bidding_zones.png}
  \end{column}

  \begin{column}{8cm}

    \begin{itemize}
    \item Bids for German electricity take place in a \alert{giant bidding zone} encompassing both Germany and Luxembourg (Austria was separated from the German bidding zone in October 2018)
    \item This means that transmission constraints are only visible to the market at the \alert{borders} to the other national zones
    \item Internal transmission constraints are \alert{ignored} - market bids are handled as if they do not exist
    \item Only KCL enforced on most borders - KVL much harder
    \end{itemize}

\end{column}
\end{columns}


\end{frame}


\begin{frame}
  \frametitle{The Problem}

  Renewables are not always located near demand centres, as in this example from Germany.



\begin{columns}[T]
  \begin{column}{6cm}
\includegraphics[width=6cm]{scigrid-load}
  \end{column}

  \begin{column}{6cm}
      % left bottom right top
  \includegraphics[trim=0 0cm 0 1cm,width=6.3cm,clip=true]{scigrid-wind}


\end{column}
\end{columns}

\end{frame}



\begin{frame}
  \frametitle{The Problem}



\begin{columns}[T]
  \begin{column}{5.5cm}
        \vspace{.5cm}
\includegraphics[width=6cm]{scigrid-loading}
  \end{column}

  \begin{column}{6cm}
    \begin{itemize}
      \item This leads to \alert{overloaded lines} in the middle of Germany, which
   cannot transport all the wind energy from North Germany to the load
   in South Germany

   \item It also overloads lines in neighbouring countries due to
     \alert{loop flows} (unplanned physical flows `according to least
     resistance' which do not correspond to traded flows)

     \item It also \alert{blocks imports and exports} with
       neighbouring countries, e.g. Denmark

    \end{itemize}

\end{column}
\end{columns}

\end{frame}

\begin{frame}
  \frametitle{Solution 1: Redispatch after energy market clearing}

  These problems are \alert{not visible} in the day-ahead electricity
  market, which treats the whole of Germany and Austria as a single
  bidding zone. It dispatches wind in North Germany as if there was no
  internal congestion...

  To ensure that the physical limits of transmission are not exceeded,
  the network operator must \alert{`re-dispatch'} power stations and
  \alert{curtail} (Einspeisemanagement) renewables to restore order.
  This is \alert{costly} (0.8 redispatch + 0.6 RE-compensation = 1.4 billion
  EUR in 2017 - although exceptional circumstances in 1st quarter) and
  results in \alert{lost CO$_2$-free generation} (5.5 TWh curtailment of RE
  and CHP in 2017).

  \alert{International redispatch} is sometimes also required
  (Multilateral Remedial Actions = MRA).

  Furthermore, there are \alert{no market incentives} to reinforce the North-South
  grid, to locate more power stations in South Germany or to
  build storage / P2X in North Germany.

\end{frame}



\begin{frame}
  \frametitle{Solution 2: Smaller bidding zones to ``see'' congested boundaries}


\begin{columns}[T]
  \begin{column}{5cm}
\includegraphics[width=6cm]{nordel.png}
  \end{column}
  \begin{column}{5cm}
    \begin{itemize}
      \item In Scandinavia they have solved this by introducing \alert{smaller bidding zones}
      \item Now congestion at the boundaries between zones is taken
        into account in the \alert{implicit auctions} of the market
      \item This is also done in Italy (again, a long country),
          where prices for small consumers are \alert{uniformised} for fairness
    \end{itemize}

\end{column}
\end{columns}


\end{frame}



\begin{frame}
  \frametitle{Solution 3: Nodal pricing}


\begin{columns}[T]
  \begin{column}{6cm}

    \vspace{.5cm}
\includegraphics[width=7cm]{lmp.png}
  \end{column}
  \begin{column}{5cm}
    \begin{itemize}
    \item The ultimate solution, as used in the US and other markets,
      is \alert{nodal pricing}, which exposes all transmission congestion
    \item Considered too complex and subject to market power to be
      used in Europe, but this is questionable...
      \item Here we see clearly why many argue for a North-South
        German split
    \end{itemize}

\end{column}
\end{columns}



\end{frame}




\begin{frame}
  \frametitle{First step: Split Germany North-South}


\begin{columns}[T]
  \begin{column}{5cm}
    \vspace{.5cm}
\includegraphics[width=5.2cm]{DE_split.png}
  \end{column}
  \begin{column}{6cm}

    \vspace{.7cm}
    \begin{itemize}

    \item Initial price difference could average up to 12 EUR/MWh
    \item Prices would converge with more network expansion
    \item Redispatch costs reduced by 39\% in 2025, 58\% in 2035 (assuming NEP 2030 transmission projects get built)
    \item Politically difficult, may require, like Italy, uniformised
      price on consumer side
    \end{itemize}
  \end{column}
\end{columns}

\source{\href{https://www.strommarkttreffen.org/2018-02_Fraunholz&Hladik_Zwei_Preiszonen_in_D.pdf}{Fraunholz \& Hladik, 2018}}
\end{frame}



\begin{frame}
  \frametitle{Solution 1.5: Flow-based market coupling}

  Flow-based market coupling can be used in zonal markets to see
  precise individual line constraints, instead of ``boxing'' the feasible space
  like ATC/NTC schemes do.

  \centering \includegraphics[width=13cm]{fbmc}

  \source{Van den Bergh, Boury, Delarue}
\end{frame}







\section{Storage Optimisation}


\begin{frame}[fragile]
  \frametitle{Storage equations}

  Now, like the network case where we add different nodes $i$
  with different loads, for storage we have to
  consider different time periods $t$.

  Label conventional generators by $s$, storage by $r$ and now minimise
  \begin{align*}
    &    \min_{\{g_{i,s,t}\},\{g_{i,r,t,\textrm{charge}}\},\{g_{i,r,t,\textrm{discharge}}\},\{f_{\ell,t}\}}\\
    &\left[  \sum_{i,s,t} o_{i,s} g_{i,s,t} +   \sum_{i,r,t} o_{i,r,\textrm{charge} }\, g_{i,r,t,\textrm{charge}} +   \sum_{i,r,t} o_{i,r,\textrm{discharge}}\, g_{i,r,t,\textrm{discharge}} \right]
  \end{align*}
  The power balance constraints are now (cf. Lecture 5) for each node $i$ and time $t$ that the demand is met either by generation, storage or network flows:
  \begin{align*}
    \sum_{s} g_{i,s,t} + \sum_{r}(g_{i,r,t,\textrm{discharge}} - g_{i,r,t,\textrm{charge}}) - d_{i,t} = \sum_\ell K_{i\ell}f_{\ell,t}  \hspace{0.5cm}\leftrightarrow\hspace{0.5cm} \l_{i,t}
  \end{align*}
  Now we have a market price $\l_{i,t}$ for each node $i$ and time $t$.
\end{frame}

\begin{frame}[fragile]
  \frametitle{Storage equations}

  We have constraints on normal generators
    \begin{align*}
        0 \leq g_{i,s,t}  \leq  G_{i,s}
  \end{align*}
    and on the storage
    \begin{align*}
    0 & \leq g_{i,r,t,\textrm{discharge}} \leq G_{i,r,\textrm{discharge}} \\
    0 & \leq g_{i,r,t,\textrm{charge}} \leq G_{i,r,\textrm{charge}}
    \end{align*}

    The energy level of the storage is given by
  \begin{align*}
    e_{i,r,t} = \eta_0e_{i,r,t-1} + \eta_1g_{i,r,t,\textrm{charge}} -  \eta_2^{-1} g_{i,r,t,\textrm{discharge}}
  \end{align*}
  and limited by
  \begin{align*}
    0 \leq e_{i,r,t} \leq E_{i,r}
  \end{align*}

\end{frame}





\begin{frame}[fragile]
  \frametitle{Idea of storage}


  Storage does `buy it low, sell it high' \alert{arbitrage}, like network, but
  in time rather than space, i.e. between cheap times (e.g. with lots
  of zero-marginal-cost renewables) and expensive times (e.g. with
  high demand, low renewables and expensive conventional generators).



\end{frame}



\begin{frame}[fragile]
  \frametitle{Storage charges at low prices, discharges at high prices}

  Simplified example from \url{https://model.energy} For Germany with only wind and hydrogen storage to meet a flat 100~MW demand.

  Average charging price (with electrolyser): 43 \euro/MWh

  Average discharging price (with turbine): 144 \euro/MWh

  %single_node_storage_bids.ipynb
    \centering \includegraphics[width=8cm]{storage_bids.pdf}



\end{frame}



\begin{frame}[fragile]
  \frametitle{Storage plus network equations}



  Finally for the flows we repeat the constraints for each time $t$.

  We have KVL for each cycle $c$ and time $t$
  \begin{align*}
    \sum_{c} C_{\ell c} x_\ell f_{\ell,t} = 0  \hspace{1cm}\leftrightarrow\hspace{1cm} \l_{c,t}
  \end{align*}
  and in addition the flows cannot overload the thermal limits, $|f_{\ell,t}| \leq F_\ell$
  \begin{align*}
    f_{\ell,t} \leq F_\ell  \hspace{1cm}\leftrightarrow\hspace{1cm} \bar{\m}_{\ell,t} \\
        - f_{\ell,t} \leq  F_\ell  \hspace{1cm}\leftrightarrow\hspace{1cm} \ubar{\m}_{\ell,t}
  \end{align*}

\end{frame}


\begin{frame}[fragile]
  \frametitle{Preview: Investment optimisation}

  Preview for next time:

  Next time we will also optimise \alert{investment} in the \alert{capacities} of generators,
  storage and network lines, to maximise \alert{long-run efficiency}.

  We will promote the capacities $G_{i,s}$, $G_{i,r,*}$, $E_{i,r}$ and
  $F_{\ell}$ to optimisation variables.

\end{frame}


\end{document}
