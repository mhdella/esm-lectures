% NB: use pdflatex to compile NOT pdftex.  Also make sure youngtab is
% there...

% converting eps graphics to pdf with ps2pdf generates way too much
% whitespace in the resulting pdf, so crop with pdfcrop
% cf. http://www.cora.nwra.com/~stockwel/rgspages/pdftips/pdftips.shtml




\documentclass[10pt,aspectratio=169,dvipsnames]{beamer}


\usetheme[color/block=transparent]{metropolis}

\usepackage[absolute,overlay]{textpos}
\usepackage{booktabs}
\usepackage[utf8]{inputenc}

\usepackage{tikz}


\usepackage[scale=2]{ccicons}

\usepackage[official]{eurosym}


%use this to add space between rows
\newcommand{\ra}[1]{\renewcommand{\arraystretch}{#1}}


\newcommand{\R}{\mathbb{R}}


\setbeamerfont{alerted text}{series=\bfseries}
\setbeamercolor{alerted text}{fg=Mahogany}
\setbeamercolor{background canvas}{bg=white}



\def\l{\lambda}
\def\m{\mu}
\def\d{\partial}
\def\cL{\mathcal{L}}
\def\co2{CO${}_2$}
\def\bra#1{\left\langle #1\right|}
\def\ket#1{\left| #1\right\rangle}
\newcommand{\braket}[2]{\langle #1 | #2 \rangle}
\newcommand{\norm}[1]{\left\| #1 \right\|}
\def\corr#1{\Big\langle #1 \Big\rangle}
\def\corrs#1{\langle #1 \rangle}


\def\mw{\text{ MW}}
\def\mwh{\text{ MWh}}
\def\emwh{\text{ \euro/MWh}}

\newcommand{\ubar}[1]{\text{\b{$#1$}}}


% for sources http://tex.stackexchange.com/questions/48473/best-way-to-give-sources-of-images-used-in-a-beamer-presentation

\setbeamercolor{framesource}{fg=gray}
\setbeamerfont{framesource}{size=\tiny}


\newcommand{\source}[1]{\begin{textblock*}{5cm}(10.5cm,8.35cm)
    \begin{beamercolorbox}[ht=0.5cm,right]{framesource}
        \usebeamerfont{framesource}\usebeamercolor[fg]{framesource} Source: {#1}
    \end{beamercolorbox}
\end{textblock*}}

\usepackage{hyperref}

\usepackage{tikz}


\usepackage[europeanresistors,americaninductors]{circuitikz}


%\usepackage[pdftex]{graphicx}


\graphicspath{{graphics/}}

\DeclareGraphicsExtensions{.pdf,.jpeg,.png,.jpg,.gif}



\def\goat#1{{\scriptsize\color{green}{[#1]}}}



\let\olditem\item
\renewcommand{\item}{%
\olditem\vspace{5pt}}

\title{Energy System Modelling\\ Summer Semester 2020, Lecture 10}
%\subtitle{---}
\author{
  {\bf Dr. Tom Brown}, \href{mailto:tom.brown@kit.edu}{tom.brown@kit.edu}, \url{https://nworbmot.org/}\\
  \emph{Karlsruhe Institute of Technology (KIT), Institute for Automation and Applied Informatics (IAI)}
}

\date{}


\titlegraphic{
  \vspace{0cm}
  \hspace{10cm}
    \includegraphics[trim=0 0cm 0 0cm,height=1.8cm,clip=true]{kit.png}

\vspace{5.1cm}

  {\footnotesize

  Unless otherwise stated, graphics and text are Copyright \copyright Tom Brown, 2020.
  Graphics and text for which no other attribution are given are licensed under a
  \href{https://creativecommons.org/licenses/by/4.0/}{Creative Commons
  Attribution 4.0 International Licence}. \ccby}
}

\begin{document}

\maketitle


\begin{frame}

  \frametitle{Table of Contents}
  \setbeamertemplate{section in toc}[sections numbered]
  \tableofcontents[hideallsubsections]
\end{frame}



\section{Cost Recovery in Long-Term Equilibrium}


\begin{frame}{Cost recovery in the long-term equilibrium}

  We will demonstrate that all players in the power network
  (generators, storage and network operators) \alert{recover their
    costs} in theory with perfect markets in long-term equilibrium and
  linear (actually convex) costs.

  If they didn't cover their costs, they would leave the market.

  If they made a profit, others would join the market and competition would reduce the profit.

  This is a direct consequence of the investment equations we considered in Lecture 9.

  We will discuss at the end why this \alert{does not work in real life}, i.e. the consequences of imperfect markets, frictions  and non-convexities.
\end{frame}


\begin{frame}{Single node with optimised capacities and dispatch}

  Suppose we have generators labelled by $s$ at a single node with \alert{marginal costs} $o_s$ arising from each unit of
  production $g_{s,t}$ and \alert{capital costs} $c_s$ that arise from fixed costs
  regardless of the rate of production (such as the investment in building
  capacity $G_s$).   For a variety of demand values $d_t$ in representative situation $t$ we optimise the total annual system costs
  \begin{equation*}
    \min_{\{g_{s,t}\},\{G_s\}}  \left[\sum_{s}c_s G_s +  \sum_{s,t} o_{s} g_{s,t} \right]
  \end{equation*}
  such that
  \begin{align*}
    \sum_s g_{s,t} & = d_t  \hspace{1cm}\leftrightarrow\hspace{1cm} \l_t \\
    - g_{s,t}  & \leq  0  \hspace{1cm}\leftrightarrow\hspace{1cm} \ubar{\m}_{s,t} \\
    g_{s,t} - G_s  & \leq 0  \hspace{1cm}\leftrightarrow\hspace{1cm} \bar{\m}_{s,t}
  \end{align*}

  We will now show using KKT that every generator exactly recovers their costs if the market price is set by $\l_t^*$, the \alert{no/zero profit rule}.
\end{frame}


\begin{frame}{Single node with optimised capacities and dispatch}

  Take the costs of generator $s$ at the optimal point:
  \begin{align*}
    c_s G_s^* +  \sum_{t}  o_{s} g_{s,t}^*
  \end{align*}

  Use stationarity for $g_{s,t}^*$
  \begin{align*}
        0 = \frac{\d \cL}{\d g_{s,t}}  = o_s - \l_t^* - \bar{\m}^*_{s,t} + \ubar{\m}^*_{s,t}
  \end{align*}
  to substitute for $o_s$ in the costs:
  \begin{align*}
    c_s G_s^* +  o_{s} \sum_{t}  g_{s,t}^* & =  c_s G_s^* + \sum_{t}( \l_t^* + \bar{\m}^*_{s,t} - \ubar{\m}^*_{s,t}) g_{s,t}^*
  \end{align*}
\end{frame}



\begin{frame}{Single node with optimised capacities and dispatch}

  Next use complementarity
  \begin{align*}
    \bar{\m}^*_{s,t}(g_{s,t}^* - G_s^*) & = 0 \\
    \ubar{\m}^*_{s,t}g_{s,t}^* & = 0
  \end{align*}
  to substitute for the terms $\m^*g_{s,t}^*$
  \begin{align*}
    c_s G_s^* +  o_{s} \sum_{t}  g_{s,t}^* & =  c_s G_s^* + \sum_{t}( \l_t^* + \bar{\m}^*_{s,t} - \ubar{\m}^*_{s,t}) g_{s,t}^* \\
    & = c_s G_s^* +  \sum_{t} \l_t^* g_{s,t}^* + \sum_t  \bar{\m}^*_{s,t} G_s^*
  \end{align*}
  Finally use stationarity for the capacity $G_s^*$
    \begin{align*}
        0 = \frac{\d \cL}{\d G_{s}}  = c_s + \sum_t \bar{\m}^*_{s,t}
  \end{align*}
  to get \alert{full cost recovery} from the market price:
  \begin{align*}
    c_s G_s^* +  o_{s} \sum_{t}  g_{s,t}^* =  \sum_{t} \l_t^* g_{s,t}^*
  \end{align*}

\end{frame}


\begin{frame}{Network of nodes with optimised capacities and dispatch}

  Suppose now we have a network of nodes $i$ connected by lines $\ell$.

  Our investment problem is now:
    \begin{equation*}
    \min_{\{g_{i,s,t}\},\{G_{i,s}\}, f_{\ell,t}, F_\ell}  \left[\sum_{i,s}c_s G_{i,s} +  \sum_{i,s,t} o_{s} g_{i,s,t}  + \sum_\ell c_\ell F_\ell \right]
  \end{equation*}
  such that
  \begin{align*}
    \sum_s g_{i,s,t} - \sum_\ell K_{i\ell}f_{\ell,t} & = d_{i,t}  \hspace{1cm}\leftrightarrow\hspace{1cm} \l_{i,t} \\
    - g_{i,s,t}  & \leq  0  \hspace{1cm}\leftrightarrow\hspace{1cm} \ubar{\m}_{i,s,t} \\
    g_{i,s,t} - G_{i,s}  & \leq 0  \hspace{1cm}\leftrightarrow\hspace{1cm} \bar{\m}_{i,s,t} \\
    f_{\ell,t} - F_\ell & \leq 0  \hspace{1cm}\leftrightarrow\hspace{1cm} \bar{\m}_{\ell,t} \\
        - f_{\ell,t} - F_\ell & \leq 0  \hspace{1cm}\leftrightarrow\hspace{1cm} \ubar{\m}_{\ell,t}
  \end{align*}


\end{frame}

\begin{frame}{Network of nodes with optimised capacities and dispatch}

  The cost recovery of the generators follows through exactly as before.

  What about the costs $c_\ell F_\ell^*$ of each transmission line?


  Use stationarity for the capacity $F_{\ell}^*$:
    \begin{align*}
        0 = \frac{\d \cL}{\d F_{\ell}}  = c_\ell   +\sum_t \bar{\m}^*_{\ell,t}  + \sum_t \ubar{\m}^*_{\ell,t}
    \end{align*}
    to get
    \begin{align*}
      c_\ell F_\ell^* = F_\ell^* \sum_t \left[ \ubar{\m}^*_{\ell,t} +  \bar{\m}^*_{\ell,t} \right]
    \end{align*}
   `At the optimal point, fixed costs equal the sum of marginal
    benefits of expanding the line at each time.'


  Next use complementarity for the flows $\bar{\m}_{\ell,t}^* ( f_{\ell,t}^* - F_\ell^*) = 0$ and $\ubar{\m}_{\ell,t}^* ( - f_{\ell,t}^* - F_\ell^*) = 0$ to get
    \begin{align*}
      c_\ell F_\ell^* =  \sum_t \left[ \bar{\m}^*_{\ell,t} -  \ubar{\m}^*_{\ell,t} \right] f_{\ell,t}^*
    \end{align*}

\end{frame}


\begin{frame}{Network of nodes with optimised capacities and dispatch}

  Finally use stationarity for each $f_{\ell,t}^*$:
    \begin{align*}
        0 = \frac{\d \cL}{\d f_{\ell,t}}  = \sum_i \l_{i,t}^*K_{i\ell}  - \bar{\m}^*_{\ell,t} + \ubar{\m}^*_{\ell,t}
  \end{align*}
    to substitute for the $\m^*$:
    \begin{align*}
      c_\ell F_\ell^*  &=  \sum_t \left[ \bar{\m}^*_{\ell,t} -  \ubar{\m}^*_{\ell,t}   \right] f_{\ell,t}^* \\
      & =\sum_t \sum_i \l_{i,t}^*K_{i\ell} f_{\ell,t}^*
    \end{align*}
    $ \sum_i \l_{i,t}^*K_{i\ell} f_{\ell,t}^*$ is nothing other than
    the \alert{congestion rent} on line $\ell$ at time $t$, i.e. the flow $f_{\ell,t}^*$ multiplied by the price difference across the line $\sum_i \l_{i,t}^*K_{i\ell}$.

    At the long-term equilibrium, the network operator covers the costs of the line exactly with the congestion rent. The optimum requires congestion at least some of the time!
\end{frame}

\begin{frame}{Storage cost recovery}

  The proof for storage is a bit grizzly, but you can find it in \href{https://arxiv.org/abs/2002.05209}{\bf\color{blue}\underline{this paper}}.

  The result is for storage unit $r$ at node $i$:
    \begin{align*}
      & c_{r,\textrm{discharge}} G_{i,r,\textrm{discharge}}^* +        c_{r,\textrm{charge}} G_{i,r,\textrm{charge}}^* +  c_{r,\textrm{energy}}  E^*_{i,r} \\
      & = \sum_t \lambda_t^* g_{i,r,t,\textrm{discharge}}^* - \sum_t \lambda_t^* g_{i,r,t,\textrm{charge}}^*
    \end{align*}

    All the costs, including the costs of the electricity to charge the storage, are recovered when the storage discharges, thereby selling its electricity to the market.

    At the equilibrium, the profits from arbitrage in the market (`buy low, sell high') exactly cover the investment costs.

    From KKT we can deduce the optimal levels at which storage should bid into the market as demand or offer as supply (more later maybe).


\end{frame}


\begin{frame}{Adding a CO2 constraint for a single node}

If we add a constraint on the total \co2 emissions
\begin{equation*}
  \sum_{s,t} \frac{\varepsilon_s}{\eta_{s}} g_{s,t} \leq \textrm{CAP}  \leftrightarrow \m_{CO2}
\end{equation*}
where $\varepsilon_s$ are the specific \co2 emissions of technology $s$ per fuel
thermal energy and $\eta_s$ is the efficiency of the generator
(i.e. the ratio between thermal energy and electrical energy). CAP
could correspond to e.g. political targets for \co2 reduction.

All that changes is stationarity for the generator
\begin{align*}
  0 =     \frac{\d \cL}{\d g_{s,t}} =   o_{s}-\l_t^* - \ubar{\m}^*_{s,t} + \bar{\m}^*_{s,t} + \m_{CO2}^* \frac{\varepsilon_s}{\eta_{s}}
\end{align*}
and now for each generator cost recovery becomes
\begin{align*}
  c_s G_s^* +  o_{s}\sum_{t} g_{s,t}^*
  & = \sum_{t} \l_t^* g_{s,t} -\m_{CO2}^* \sum_{t}  \frac{\varepsilon_s}{\eta_{s}} g_{s,t}^*
\end{align*}

This shows nicely the duality for exchanging the CO2 constraint for a CO2 price $o_{s} \to o_{s} + \m^*_{CO2} \frac{\varepsilon_s}{\eta_{s}} $.

\end{frame}



\begin{frame}{Introduction to Lagrangian Relaxation}

  This switching between costs and constraints is a special case of \alert{Lagrangian relaxation}.

  Consider the optimisation problem:
  \begin{equation*}
  \max_{x} f(x)
\end{equation*}
[$x = (x_1, \dots x_k)$] subject to some \alert{constraints} within $\R^k$:
\begin{align*}
  g_i(x) & = c_i \hspace{1cm}\leftrightarrow\hspace{1cm} \l_i \hspace{1cm} i = 1,\dots n \\
  h_0(x) & = d_0  \hspace{1cm}\leftrightarrow\hspace{1cm} \m_0 \\
  h_j(x) & \leq d_j \hspace{1cm}\leftrightarrow\hspace{1cm} \m_j \hspace{1cm} j = 1,\dots m
\end{align*}

\end{frame}



\begin{frame}{Introduction to Lagrangian Relaxation}

  Now consider the related problem where $\tilde\mu_0$ is fixed to a constant:
  \begin{equation*}
  \max_{x} f(x) - \tilde\mu_0 (h_0(x) - d_0)
\end{equation*}
[$x = (x_1, \dots x_k)$] subject to some \alert{constraints} within $\R^k$:
\begin{align*}
  g_i(x) & = c_i \hspace{1cm}\leftrightarrow\hspace{1cm} \l_i \hspace{1cm} i = 1,\dots n \\
  h_j(x) & \leq d_j \hspace{1cm}\leftrightarrow\hspace{1cm} \m_j \hspace{1cm} j = 1,\dots m
\end{align*}

We have \alert{relaxed} the problem by removing one of the constraints.

You can show that the new problem has the same solution as the old $(x^*,\l^*,\m^*)$ if we fix the constant $\tilde\mu_0 = \mu_0^*$ by comparing the KKT stationarity constraints of the two problems.

We have lifted the constraint into the objective function, where it penalises solutions with $h_0(x) > d_0$.

In general, if we don't know $\mu_0^*$ beforehand, we can iteratively solve to find it.

Often it can be easier to solve the relaxed problem.


\end{frame}


\begin{frame}{Fundamental Welfare Theorem is Lagrangian Relaxation}

  Consider the maximisation of total welfare:
  \begin{align*}
    \max_{\{d_b\}, \{g_s\}} f(\{d_b\}, \{g_s\}) = \left[ \sum_b U_b (d_b)  -  \sum_s C_s (g_s) \right]
  \end{align*}
  subject to the balance constraint:
  \begin{align*}
    g(\{d_b\}, \{g_s\}) = \sum_b d_b -  \sum_s g_s  = 0 \hspace{1cm} \leftrightarrow \hspace{1cm} \l
  \end{align*}
  Now let's relax the constraint:
  \begin{align*}
    \max_{\{d_b\}, \{g_s\}} f(\{d_b\}, \{g_s\}) = \left[ \sum_b U_b (d_b)  -  \sum_s C_s (g_s) - \tilde\lambda ( \sum_b d_b -  \sum_s g_s) \right]
  \end{align*}
  This problem is \alert{separable} and \alert{decomposes} into separate problems for each $d_b$:
  \begin{equation*}
    \max_{d_b} \left[ U_b(d_b) - \tilde\lambda d_b\right]
  \end{equation*}
  and for each $g_s$:
  \begin{equation*}
    \max_{g_s} \left[ \tilde\lambda g_s  -  C_s(g_s)\right]
  \end{equation*}

\end{frame}

\begin{frame}{Grit in the machine for generation 1/2}

  Several factors make this theoretical picture quite different in reality:
  \begin{itemize}
  \item Generation investment is \alert{lumpy} i.e. you can often only
    build power stations in e.g. 500~MW blocks, not at any size.
  \item Some older generators have \alert{sunk costs}, i.e. costs which have been incurred once and cannot be recovered, which alters their behaviour (i.e. the capital cost term $c_s$ is not evenly distributed across all hours)
  \item Returns on scale in building plant are not taken into account (this would be a non-convexity; we did everything linear)
  \item Site-specific concerns ignored (e.g. lignite might need to be near a mine and have limited capacity)
  \item Variability of production for wind/solar ignored
    \item There is considerable  uncertainty given load/weather conditions during a year, which makes investment risky; economic downturns reduce electricity demand
  \end{itemize}

\end{frame}



\begin{frame}{Grit in the machine for generation 2/2}

  Several factors make this theoretical picture quite different in reality:
  \begin{itemize}
    \item Fuel cost fluctuations, building delays which cost money
  \item Risks from third-parties:  Changing costs of other generators, political risks (\co2 taxes,  Atomausstieg, subsidies for renewables, price caps)
  \item Political or administrative constraints on wholesale energy
    prices may prevent prices from rising high enough for long enough
    to justify generation investment (``Missing Money Problem'')
  \item  Lead-in time for planning and building, behaviour of others, boom-and-bust investment cycles resulting from periods of under- and over-investment in capacity
    \item Exercise of \alert{market power}
  \end{itemize}

\end{frame}


\begin{frame}{Episodes of High Prices are an Essential Part of an Energy-Only Market}

  In an energy-only market (in which generators are only compensated
  for the energy they produce), the wholesale spot price must at times
  be higher than the variable cost of the highest-variable-cost
  generating unit in the market. Episodes of high prices and/ or price
  spikes are not in themselves evidence of market power or evidence of
  market failure.

  However, there may be political or administrative restrictions on
  prices going to very high levels (i.e. consumer protection, concerns about market abuse).




\end{frame}


\begin{frame}{Today's market does not have episodes of very high prices}

  This makes it hard for e.g. gas generators to make back their
  costs. Day ahead spot market prices in 2016 in Germany-Austria bidding zone:

  \centering
  \includegraphics[width=8cm]{germany-2016-price-duration}

  \raggedright
  Gas generators can bid into other markets, such as the intra-day or
  reserve power markets, or provide redispatch services.



\end{frame}


\begin{frame}{Market prices from highly renewable simulations}

  In our simulations for high renewable penetrations, the theory does
  however work:

  \centering
  \includegraphics[width=8.5cm]{europe-future-price-duration}

  \raggedright
  Prices are zero around a quarter of the time, but spike above 10,000 \euro/MWh in some hours.

\end{frame}

\begin{frame}{Price cap}

  Some markets implement a maximum market price cap (MPC), which may be below the Value of Lost Load (VoLL)
  ($V$ for the inelastic case).

  In the Eastern Australian National Electricity Market (NEM), a MPC
  of A\$13,800/MWh (\euro~9,300/MWh) for the 2015-2016 financial year
  is set, corresponding to the price automatically triggered when AEMO
  directs network service providers to interrupt customer supply in
  order to keep supply and demand in the system in balance.

  \centering

\begin{columns}[T]
\begin{column}{6.5cm}



  \includegraphics[width=6.5cm]{price-cap}
\end{column}
\begin{column}{5cm}


  This can introduce distortions which make it difficult for some generators to recover costs.

\end{column}

\end{columns}


  \source{Biggar and Hesamzadeh, 2014}
\end{frame}



\begin{frame}{Europe: Capacity Markets in Some Countries}

  CRM = Capacity Remuneration Mechanisms (CRM): status June 2014
  \centering
  \includegraphics[width=10.5cm]{cap_markets.png}


  \source{\href{http://www.mdpi.com/1996-1073/8/6/5198metrics}{Ellenbeck et al, 2014}}

\end{frame}

\begin{frame}{Features of Transmission Investment 1/2}


  \begin{enumerate}
  \item \alert{Rationale for transmission}: Load and generation do not coincide in location at all times, so electricity must be transported for some of the time.
  \item \alert{Transmission is a natural monopoly}: Like railways or water provision, it is unlikely that a parallel electricity network would be built, given cost and limits on installing infrastructure due to space and public acceptance. Natural monopolies require \alert{regulation}.
  \item \alert{Transmission is a capital-intensive business}:
    Transmitting electric power securely and efficiently over long
    distances requires large amounts of equipment (lines,
    transformers, etc.) which dominate costs compared to the operating
    costs of the grid. Making good investment decisions is thus the
    most important aspect of running a transmission company.
  \end{enumerate}



\end{frame}


\begin{frame}{Features of Transmission Investment 2/2}


  \begin{enumerate}
  \item \alert{Transmission assets have a long life}: Most
    transmission equipment is designed for an expected life ranging
    from 20 to 40 years or even longer (up to 60-80 years). A lot can
    change over this time, such as load behaviour and generation costs
    and composition.
  \item \alert{Transmission investments are irreversible}: Once a transmission line has been built, it
  cannot be redeployed in another location where it could be used more profitably.
  \item \alert{Transmission investments are lumpy}: Manufacturers sell transmission equipment in
    only a small number of standardized voltage and MVA ratings. It is therefore often not
    possible to build a transmission facility whose rating exactly matches the need.
    \item \alert{Economies of scale}: Transmission investment more
      proportional to length (costs of rights of way, terrain, towers,
      which dominate costs) than to power rating (which depends only
      on conductoring, which is cheap).

  \end{enumerate}


\end{frame}



\section{Integrating Renewables in Power Markets}


\begin{frame}{Characteristics of Renewables}

  \begin{itemize}
  \item \alert{Variability}: Their production depends on weather (wind speeds for wind, insolation for solar and precipitation for hydroelectricity)
  \item \alert{No Upwards Controllability}: Variable Renewable Energy (VRE) like wind and solar can only reduce their output; raising is hard
  \item \alert{No Long-Term Forecastability}: Although short-term forecasting is improving steadily
  \item \alert{Low Marginal Cost} (no fuel costs)
  \item \alert{High Capital Cost}
  \item \alert{No Direct Carbon Dioxide Emissions} (but some indirect ones from manufacturing)
  \item \alert{Small unit size} (wind turbine is 2-3~MW; coal/nuclear is 1000~MW)
  \item \alert{Somewhat Decentralised Distribution} for some VRE (e.g. solar panels on household rooves); offshore is however very centralised
    \item \alert{Provision of system services}: Increasing
  \end{itemize}

\end{frame}

\begin{frame}{RE Levelised Cost already approaching fossil fuels}

  \centering
  \includegraphics[width=10.5cm]{irena_costs}

  \source{\href{http://www.irena.org/DocumentDownloads/Publications/IRENA_RE_Power_Costs_2014_report.pdf}{IRENA Renewable Generation Costs}}
\end{frame}


\begin{frame}{RE Forecasting}

  Just like the weather on which it depends, Variable RE (wind and
  solar) production can be forecast in advance.   (Shaded area is the uncertainty.)

  \centering
  \includegraphics[width=9.5cm]{wind_farm_forecast_2}
\end{frame}



\begin{frame}{RE Forecasting}

  Like the weather, the forecast in the short-term (e.g. day ahead) is
  fairly reliable, particularly for wind, but for several days ahead
  it is less useful. In addition, it is subject to more uncertainty
  than the load. For example, fog and mist is very local, hard to
  predict, and has a big impact on solar power production.


  This makes scheduling more challenging and has led to the introduction of
  more regular auctions in the intraday market.

  Forecasting has also become a big business.

\end{frame}


\begin{frame}{Effect on effective `residual' load curve}

  Since RE often have priority feed-in (i.e. network operators are obliged to take their power), we often subtract the RE production from the load to get the \alert{residual load}, plotted here as a demand-duration-curve.

  \centering
  \includegraphics[width=10cm]{residual-load}

  \source{Biggar and Hesamzadeh, 2014}
\end{frame}


\begin{frame}{Residual load curve and screening curve}



\begin{columns}[T]
\begin{column}{6.5cm}



  \centering
    \begin{tikzpicture}
\node[anchor=south west,inner sep=0] (image) at (0,0) {\includegraphics[width=5.4cm]{residual-screening_curve-clean}};
\draw (0.3,4.9) node{$c_2$};
\draw (0.3,6.5) node{$c_1$};
  \end{tikzpicture}

\end{column}
\begin{column}{6cm}

  The residual load must be met by conventional generators.

  The changed duration curve interacts differently with the screening curve, so that we may require less baseload generation and peaking plant and more load shedding, depending on the shape of the curve.

  In some markets, there is increased demand for medium-peaking plant.

\end{column}



\end{columns}


  \centering

  \source{Biggar and Hesamzadeh, 2014}
\end{frame}


\begin{frame}
  \frametitle{Effect of varying renewables: fixed demand, no wind}

  \centering
  \includegraphics[width=9cm]{demand-supply-no_wind}

\end{frame}



\begin{frame}
  \frametitle{Effect of varying renewables: fixed demand, 35~GW wind}

  \centering
  \includegraphics[width=9cm]{demand-supply-with_wind}

\end{frame}


\begin{frame}
  \frametitle{Spot market price development}

  As a result of so much zero-marginal-cost renewable feed-in, spot
  market prices steadily decreased until 2016 (but since went up again):

  \centering
  \includegraphics[width=11cm]{price_development}

  \source{Agora Energiewende}
\end{frame}



\begin{frame}
  \frametitle{Merit Order Effect}

  To summarise:
  \begin{itemize}
  \item Renewables have zero marginal cost
  \item As a result they enter at the bottom of the merit order, reducing the price at which the market clears
  \item This pushes non-CHP gas and hard coal out of the market
  \item This is unfortunate, because among the fossil fuels, gas and
    hard coal are the most flexible and produce the \emph{lowest} \co2
    per MWh
  \item It also massively reduces the profits that nuclear and brown coal make
  \item Will there be enough backup power plants for times with no wind/solar?
  \end{itemize}

  This has led to lots of political tension...

\end{frame}

\begin{frame}{Market value}

  VRE have the property that they cannibalise their own market, by
  pushing down prices when lots of other VRE are producing.

  We define the \alert{market value} of a technology by the average
  market price it receives when it produces, i.e.

  \begin{equation*}
    MV_s = \frac{\sum_{t} \l^*_t g_{s,t}}{\sum_{t} g_{s,t}}
  \end{equation*}

  We can compare this to the average market price,
    defined either as the simple average $\frac{1}{T} \sum_t \l^*_t$ or the demand-weighted average $ \frac{\sum_{t} \l^*_td_t}{\sum_{t}d_t}$.

\end{frame}



\begin{frame}{Historic market values in Germany}

  \centering

  \includegraphics[width=11cm]{market_value-historical}



  \source{\href{https://doi.org/10.1016/j.eneco.2013.02.004}{Lion Hirth, 2013}}
\end{frame}

\begin{frame}{Market value at higher shares}


  At low shares of VRE the market value may be higher than the average market price (because for example, PV produces a midday when prices are higher than average), but as VRE share increases the market value goes down.

\begin{columns}[T]
\begin{column}{6.5cm}



  \centering

  \includegraphics[width=6cm]{market_value_us}


\end{column}
\begin{column}{4cm}

  The effect is particularly severe for PV, since the production is highly correlated; for wind smoothing prevents a steeper drop off. The bigger the catchment area, the longer wind preserves its market value.

  \source{Mills \& Wiser, 2014}
\end{column}



\end{columns}


\end{frame}



\begin{frame}{Market value mitigation}

To halt the drop in market value (and hence revenue for wind and
solar) we can use networks to do price arbitrage in space, storage to
do arbitrage in time, or introduce CO2 prices that push up the prices in times when fossil fuel
plants are running.

\begin{columns}[T]
\begin{column}{5.5cm}



  \centering

  \includegraphics[width=5cm]{market_value-networks}


\end{column}
\begin{column}{5.5cm}


  \centering

  \includegraphics[width=5cm]{market_value-storage}
\end{column}

\end{columns}


  \source{\href{https://doi.org/10.1016/j.eneco.2013.02.004}{Lion Hirth, 2013}}

\end{frame}


\begin{frame}
  \frametitle{Market value from our 95\% renewable simulations}
  \begin{columns}[T]
    \begin{column}{8.3cm}
      \vspace{1cm}
\centering
  \includegraphics[width=8.3cm]{pre-7-market_value}
  %\includegraphics[width=6cm]{pre-2-costs_de_storage}
\end{column}
\begin{column}{4cm}
  \begin{itemize}
%\item   %Big reduction in curtailment
  \item Storage charges at low market prices and dispatches at high
    prices.
  \item Dispatchable power sources take advantage of high prices.
  \item Variable renewables get lower prices, but saved by storage, networks and high CO2 price.
  \end{itemize}
\end{column}
\end{columns}


\end{frame}

\begin{frame}
  \frametitle{Relation of LCOE to market value}

  From the first section we had for a perfect market in long-term equilibrium that all costs are recovered from market revenue:
  \begin{align*}
    c_s G_s^* +  o_{s} \sum_{t}  g_{s,t}^* =  \sum_{t} \l_t^* g_{s,t}^*
  \end{align*}
  If we divide both sides by the total yearly generation $\sum_{t} g_{s,t}^*$ then we get:
  \begin{align*}
    \frac{c_s G_s^* +  o_{s} \sum_{t}  g_{s,t}^*}{\sum_{t} g_{s,t}^*} =  \frac{\sum_{t} \l_t^* g_{s,t}^*}{\sum_{t} g_{s,t}^*}
  \end{align*}
  This is none other than the identity between the LCOE and market value:
  \begin{align*}
    LCOE = MV
  \end{align*}
  This \emph{only} holds in a perfect equilibrium. I.e. the
  equilibrium is found by increasing the penetration until the market
  value equals the LCOE.

  In reality the market is far from equilibrium: subsidies support
  technologies (with a longer-term view of pushing them down the
  learning curve), there are sunk costs for existing plants, excess
  capacity supported outside the energy-only market, etc.
\end{frame}


\end{document}
