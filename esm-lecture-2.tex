% NB: use pdflatex to compile NOT pdftex.  Also make sure youngtab is
% there...

% converting eps graphics to pdf with ps2pdf generates way too much
% whitespace in the resulting pdf, so crop with pdfcrop
% cf. http://www.cora.nwra.com/~stockwel/rgspages/pdftips/pdftips.shtml




\documentclass[10pt,aspectratio=169,dvipsnames]{beamer}
\usetheme[color/block=transparent]{metropolis}

\usepackage[absolute,overlay]{textpos}
\usepackage{booktabs}
\usepackage[utf8]{inputenc}


\usepackage[scale=2]{ccicons}

\usepackage[official]{eurosym}

%use this to add space between rows
\newcommand{\ra}[1]{\renewcommand{\arraystretch}{#1}}


\setbeamerfont{alerted text}{series=\bfseries}
\setbeamercolor{alerted text}{fg=Mahogany}
\setbeamercolor{background canvas}{bg=white}


\newcommand{\R}{\mathbb{R}}

\def\l{\lambda}
\def\m{\mu}
\def\d{\partial}
\def\cL{\mathcal{L}}
\def\co2{CO${}_2$}



% for sources http://tex.stackexchange.com/questions/48473/best-way-to-give-sources-of-images-used-in-a-beamer-presentation

\setbeamercolor{framesource}{fg=gray}
\setbeamerfont{framesource}{size=\tiny}


\newcommand{\source}[1]{\begin{textblock*}{5cm}(10.5cm,8.35cm)
    \begin{beamercolorbox}[ht=0.5cm,right]{framesource}
        \usebeamerfont{framesource}\usebeamercolor[fg]{framesource} Source: {#1}
    \end{beamercolorbox}
\end{textblock*}}

\usepackage{hyperref}


%\usepackage[pdftex]{graphicx}


\graphicspath{{graphics/}}

\DeclareGraphicsExtensions{.pdf,.jpeg,.png,.jpg}



\def\goat#1{{\scriptsize\color{green}{[#1]}}}



\let\olditem\item
\renewcommand{\item}{%
\olditem\vspace{5pt}}

\title{Energy System Modelling\\ Summer Semester 2020, Lecture 2}
%\subtitle{---}
\author{
  {\bf Dr. Tom Brown}, \href{mailto:tom.brown@kit.edu}{tom.brown@kit.edu}, \url{https://nworbmot.org/}\\
  \emph{Karlsruhe Institute of Technology (KIT), Institute for Automation and Applied Informatics (IAI)}
}

\date{}


\titlegraphic{%
  \vspace{0cm}
  \hspace{10cm}
    \includegraphics[trim=0 0cm 0 0cm,height=1.8cm,clip=true]{kit.png}

\vspace{5.1cm}

  {\footnotesize

  Unless otherwise stated, graphics and text are Copyright \copyright Tom Brown, 2020.
  Graphics and text for which no other attribution are given are licensed under a
  \href{https://creativecommons.org/licenses/by/4.0/}{Creative Commons
  Attribution 4.0 International Licence}. \ccby}

}

\begin{document}

\maketitle


\begin{frame}

  \frametitle{Table of Contents}
  \setbeamertemplate{section in toc}[sections numbered]
  \tableofcontents[hideallsubsections]
\end{frame}


\section{Variable Renewable Energy (VRE)}



\begin{frame}
  \frametitle{Solar time series}

  Unlike the load, the solar feed-in is much more variable, dropping to zero and not reaching full output (when aggregated over all of Germany).


  \centering
  \includegraphics[width=13cm]{DE-solar-H}

\end{frame}


\begin{frame}
  \frametitle{Solar time series: weekly}

  If we take a weekly average we see higher solar in the summer.

  \centering
  \includegraphics[width=13cm]{DE-solar-W}

\end{frame}



\begin{frame}
  \frametitle{Solar duration curve}



  \centering
  \includegraphics[width=13cm]{DE-solar-duration}

\end{frame}




\begin{frame}
  \frametitle{Solar density function}

  \centering
  \includegraphics[width=13cm]{DE-solar-density}

\end{frame}




\begin{frame}
  \frametitle{Solar spectrum}

  If we Fourier transform, the \alert{seasonal} and \alert{daily} patterns become visible.

  \centering
  \includegraphics[width=13cm]{DE-solar-spectrum}

\end{frame}



\begin{frame}
  \frametitle{Wind time series}

  Wind is variable, like solar, but the variations are on different time scales. It drops close to zero and rarely reaches full output (when aggregated over all of Germany).

  \centering
  \includegraphics[width=13cm]{DE-onwind-H}

\end{frame}


\begin{frame}
  \frametitle{Wind time series: weekly}

  If we take a weekly average we see higher wind in the winter and
  some periodic patterns over 2-3 weeks (\alert{synoptic scale}).

  \centering
  \includegraphics[width=13cm]{DE-onwind-W}

\end{frame}



\begin{frame}
  \frametitle{Wind duration curve}



  \centering
  \includegraphics[width=13cm]{DE-onwind-duration}

\end{frame}




\begin{frame}
  \frametitle{Wind density function}

  \centering
  \includegraphics[width=13cm]{DE-onwind-density}

\end{frame}




\begin{frame}
  \frametitle{Wind spectrum}

  If we Fourier transform, the \alert{seasonal}, \alert{synoptic} (2-3 weeks) and \alert{daily} patterns become visible.

  \centering
  \includegraphics[width=13cm]{DE-onwind-spectrum}

\end{frame}


\section{Balancing a single country}



\begin{frame}
  \frametitle{Power mismatch}

  Suppose we now try and cover the electrical demand with the
  generation from wind and solar.

  How much wind do we need? We have three time series:
  \begin{itemize}
  \item $\{ d_t\}, d_t \in \R$ the load (varying between 35 GW and 80 GW)
  \item $\{ w_t\}, w_t \in [0,1]$ the wind availability (how much a 1 MW wind turbine produces)
  \item $\{ s_t\}, s_t \in [0,1]$ the solar availability  (how much a 1 MW solar turbine produces)
  \end{itemize}

  We try $W$ MW of wind and $S$ MW of solar. Now the effective \alert{residual load} or \alert{mismatch} is
  \begin{equation*}
    m_t = d_t - Ww_t - Ss_t
  \end{equation*}

  We choose $W$ and $S$ such that on \alert{average} we cover all the load
  \begin{equation*}
    \langle m_t \rangle = 0
  \end{equation*}
  and so that the 70\% of the energy comes from wind and 30\% from solar ($W = 147$ GW and $S = 135$ GW).

\end{frame}




\begin{frame}
  \frametitle{Mismatch time series}


  \centering
  \includegraphics[width=13cm]{DE-mismatch-H}

\end{frame}



\begin{frame}
  \frametitle{Mismatch duration curve}



  \centering
  \includegraphics[width=13cm]{DE-mismatch-duration}

\end{frame}




\begin{frame}
  \frametitle{Mismatch density function}

  \centering
  \includegraphics[width=13cm]{DE-mismatch-density}

\end{frame}




\begin{frame}
  \frametitle{Mismatch spectrum}

  If we Fourier transform, the synoptic (from wind) and daily patterns (from demand and solar) become visible. Seasonal variations appear to cancel out.

  \centering
  \includegraphics[width=13cm]{DE-mismatch-spectrum}

\end{frame}



\begin{frame}
  \frametitle{How to deal with the mismatch?}

  The problem is that
    \begin{equation*}
    \langle m_t \rangle = 0
  \end{equation*}
    is not good enough! We need to meet the demand in every single hour.

    This means:
    \begin{itemize}
      \item If $m_t > 0$, i.e. we have unmet demand, then we need
        backup generation from \alert{dispatchable} sources
        e.g. hydroelectricity reservoirs, fossil/biomass fuels.
      \item If $m_t < 0$, i.e. we have over-supply, then we have to
        shed / spill / \alert{curtail} the renewable energy.
    \end{itemize}


\end{frame}


\begin{frame}
  \frametitle{Mismatch}


  \centering
  \includegraphics[width=13cm]{mismatch-2011-03-01-2011-03-31}


\end{frame}

\begin{frame}
  \frametitle{Mismatch}


  \centering
  \includegraphics[width=13cm]{mismatch-2011-07-01-2011-07-31}


\end{frame}

\begin{frame}
  \frametitle{Mismatch}


  \centering
  \includegraphics[width=13cm]{mismatch-2011-12-01-2011-12-31}


\end{frame}



\begin{frame}
  \frametitle{Mismatch duration curve}


  \centering
  \includegraphics[width=13cm]{mismatch-duration}


\end{frame}



\begin{frame}
  \frametitle{What to do?}

  Backup energy costs money and may also cause CO${}_2$ emissions.

  Curtailing renewable energy is also a waste.

  We'll look in the next lectures at \alert{four solutions}:
  \begin{enumerate}
  \item \alert{Smoothing} stochastic variations of renewable feed-in \alert{over continental areas}, e.g. the whole of Europe.
  \item Using \alert{electricity storage} to shift energy from times of surplus to times of deficit.
  \item Shifting demand to different times, when renewables are abundant, i.e. \alert{demand-side management} (DSM).
    \item Consuming the electricity in \alert{other sectors}, e.g. transport or heating, where there are further possibilities for DSM (battery electric vehicles, heat pumps)  and cheap storage possibilities (e.g. thermal storage or power-to-gas as hydrogen or methane).
  \end{enumerate}



\end{frame}



\end{document}
