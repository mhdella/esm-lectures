% NB: use pdflatex to compile NOT pdftex.  Also make sure youngtab is
% there...

% converting eps graphics to pdf with ps2pdf generates way too much
% whitespace in the resulting pdf, so crop with pdfcrop
% cf. http://www.cora.nwra.com/~stockwel/rgspages/pdftips/pdftips.shtml




\documentclass[10pt,aspectratio=169,dvipsnames]{beamer}
\usetheme[color/block=transparent]{metropolis}

\usepackage[absolute,overlay]{textpos}
\usepackage{booktabs}
\usepackage[utf8]{inputenc}


\usepackage[scale=2]{ccicons}

\usepackage[official]{eurosym}


\usepackage{tikz}
\usetikzlibrary{arrows.meta}

\usepackage{xcolor}

\makeatletter
\def\mathcolor#1#{\@mathcolor{#1}}
\def\@mathcolor#1#2#3{%
  \protect\leavevmode
  \begingroup
    \color#1{#2}#3%
  \endgroup
}
\makeatother

%use this to add space between rows
\newcommand{\ra}[1]{\renewcommand{\arraystretch}{#1}}


\setbeamerfont{alerted text}{series=\bfseries}
\setbeamercolor{alerted text}{fg=Mahogany}
\setbeamercolor{background canvas}{bg=white}


\newcommand{\R}{\mathbb{R}}

\def\l{\lambda}
\def\m{\mu}
\def\d{\partial}
\def\cL{\mathcal{L}}
\def\co2{CO${}_2$}



% for sources http://tex.stackexchange.com/questions/48473/best-way-to-give-sources-of-images-used-in-a-beamer-presentation

\setbeamercolor{framesource}{fg=gray}
\setbeamerfont{framesource}{size=\tiny}


\newcommand{\source}[1]{\begin{textblock*}{5cm}(10.5cm,8.35cm)
    \begin{beamercolorbox}[ht=0.5cm,right]{framesource}
        \usebeamerfont{framesource}\usebeamercolor[fg]{framesource} Source: {#1}
    \end{beamercolorbox}
\end{textblock*}}

\usepackage{hyperref}


%\usepackage[pdftex]{graphicx}


\graphicspath{{graphics/}}

\DeclareGraphicsExtensions{.pdf,.jpeg,.png,.jpg}



\def\goat#1{{\scriptsize\color{green}{[#1]}}}


\newcommand{\ubar}[1]{\text{\b{$#1$}}}

\let\olditem\item
\renewcommand{\item}{%
\olditem\vspace{5pt}}

\title{Energy System Modelling\\ Summer Semester 2020, Lecture 16}
%\subtitle{---}
\author{
  {\bf Dr. Tom Brown}, \href{mailto:tom.brown@kit.edu}{tom.brown@kit.edu}, \url{https://nworbmot.org/}\\
  \emph{Karlsruhe Institute of Technology (KIT), Institute for Automation and Applied Informatics (IAI)}
}

\date{}


\titlegraphic{
  \vspace{0cm}
  \hspace{10cm}
    \includegraphics[trim=0 0cm 0 0cm,height=1.8cm,clip=true]{kit.png}

\vspace{5.1cm}

  {\footnotesize

  Unless otherwise stated, graphics and text are Copyright \copyright Tom Brown, 2020.
  Graphics and text for which no other attribution are given are licensed under a
  \href{https://creativecommons.org/licenses/by/4.0/}{Creative Commons
  Attribution 4.0 International Licence}. \ccby}
}

\begin{document}

\maketitle


\begin{frame}

  \frametitle{Table of Contents}
  \setbeamertemplate{section in toc}[sections numbered]
  \tableofcontents[hideallsubsections]
\end{frame}


\section{The World is Not a Perfect Optimization Model}


\begin{frame}
  \frametitle{We should be skeptical about models and modellers}
\begin{columns}[T]
  \begin{column}{7.5cm}

    \vspace{.2cm}
    \centering
    \includegraphics[width=7cm]{2019-01-10-IEEFA-EIA-coal-all-consumption-forecasts-470-x-395-v2-768x646}
  \end{column}
  \begin{column}{7.5cm}

    \vspace{.2cm}
    \centering
    \includegraphics[width=7cm]{auke.jpg}
  \end{column}
\end{columns}
\end{frame}

\begin{frame}
  \frametitle{We should be skeptical about models and modellers}

\begin{columns}[T]
  \begin{column}{9cm}

    \vspace{.2cm}
    \centering
    \includegraphics[width=9.5cm]{hubbert.png}
  \end{column}
  \begin{column}{5cm}

    \vspace{.2cm}
    \begin{itemize}
    \item Possible scenario projected from 1956 by US geologist M. King Hubbert
    \item Oil production in the US did indeed peak in the 1970s, but returned to peak height in last decade thanks to shale oil extraction with fracking
    \item Nuclear expanded but plateaued
      \item \alert{What might we be getting wrong in 2020?}
    \end{itemize}
  \end{column}
\end{columns}

\source{\href{http://www.energycrisis.com/Hubbert/1956/1956.pdf}{Hubbert, 1956}}
\end{frame}


\begin{frame}
  \frametitle{We should be skeptical about models and modellers}

  Models can:
  \vspace{-.3cm}
    \begin{itemize}
    \item \alert{under- or overestimate rates of change} (e.g. under: PV uptake, over: onshore wind in UK/Germany/Netherlands)
    \item \alert{underestimate social factors} (e.g. concern about nuclear / transmission / wind)
    \item \alert{extrapolate based on uncertain data} (e.g. oil reserves, learning curves for PV)
    \item \alert{focus on easy-to-solve rather than policy-relevant problems} (e.g. most research)
    \item \alert{neglect uncertainty} (e.g. in short-term due to weather forecasts, or in long-term due to cost, political uncertainty and technological development)
    \item \alert{neglect need for robustness} (e.g. securing energy system against contingencies, attack)
    \item \alert{neglect complex interactions of markets and incentive structures} (e.g. abuse of market power, non-linearities not represented in models, lumpiness, etc.)
    \item \alert{neglect non-linearities and non-convexities} (e.g. power flow, or also learning curves, behavioural effects, perverse local optima, many, many more)
    \end{itemize}

\end{frame}


\section{Robustness to Different Weather Years}


\begin{frame}
  \frametitle{Different Weather Years}

  Many of the simulations we looked at in this course, and many in the literature, used single weather years to determine optimal investments.

  This is problematic since:
  \begin{itemize}
  \item Weather changes from year to year
  \item There are decadal variations of wind
  \item Demand changes (particularly space heating demand during cold years)
  \end{itemize}

  But computing investments against 30 years of data (262,800 hours) is not feasible.

\end{frame}


\begin{frame}
  \frametitle{Different Weather Years}

  If we use different weather years to optimize sector-coupled European model with net-zero CO$_2$ emissions (including industry) we see broadly stable technology choices but variations in total system costs of up to 20\%.  NB: In real world cannot reoptimize investment every year!

  \centering
    \includegraphics[width=13cm]{lin-total_costs.png}

    \source{Lin Yang MA thesis}

\end{frame}


\begin{frame}
  \frametitle{Different Weather Years}

  Biggest changes are driven by space heating demand. Cold years (like 2010) are more expensive.

  \centering
    \includegraphics[width=14cm]{lin-heating.png}


    \source{Lin Yang MA thesis}

\end{frame}


\begin{frame}
  \frametitle{Different Weather Years}

  Optimal technology investments do not change dramatically from year to year.

  \centering
    \includegraphics[width=14cm]{lin-generators.png}


    \source{Lin Yang MA thesis}

\end{frame}


\begin{frame}
  \frametitle{Different Weather Years}

  If we fix the optimal technology investments based on the weather of one year ($y$-axis), then run the dispatch over all 30 years (900 simulations in total), we can assess average curtailment and load-shedding. Using coldest year 2010 gives low load-shedding but high curtailment.

  \centering
    \includegraphics[width=14cm]{lin-run_30.png}

    \source{Lin Yang MA thesis}

\end{frame}


\begin{frame}
  \frametitle{Using 2010 investments}

  Using coldest year 2010 guarantees virtually no load-shedding in entire 30 years, but leads to excess energy in most years.
  Better to store excess energy from warmer years (e.g. chemically).

  \centering
    \includegraphics[width=14cm]{lin-use_2010.png}

    \source{Lin Yang MA thesis}

\end{frame}



\section{Robustness to Climate Change}

\section{Myopic Foresight for Weather Uncertainty}

%\section{Myopic Foresight for Multi-Decade Investment}

\section{Cost and Political Uncertainty}


\begin{frame}
  \frametitle{Sensitivity of Optimisation to Cost, Weather Data and Policy Constraints}

  See Schlachtberger et al, `Cost optimal scenarios of a future highly renewable European
  electricity system: Exploring the influence of weather data, cost
  parameters and policy constraints,' 2018,
  \url{https://arxiv.org/abs/1803.09711}

\end{frame}

\section{Near-Optimal Energy Systems}



\begin{frame}
  \frametitle{Large Space of Near-Optimal Energy Systems}

  There is a \alert{large degeneracy} of different possible energy systems close to the optimum.

  \vspace{.2cm}

  % epsilon constraint
  \centering
  \includegraphics[width=12cm]{graphics/sketch.png}

  \source{\href{https://arxiv.org/abs/1910.01891}{Neumann \& Brown, 2020}}

\end{frame}



\begin{frame}
  \frametitle{Example: 100\% renewable electricity system for Europe}

\begin{columns}[T]
  \begin{column}{7cm}
    Capacity expansion in optimum:

    \includegraphics[width=7cm]{no-optimum.png}
  \end{column}
  \begin{column}{7cm}

    $\varepsilon =$ 10\% above optimum, minimise new grid:

    \includegraphics[width=7cm]{no-trans-10.png}
  \end{column}
\end{columns}



  \source{\href{https://arxiv.org/abs/1910.01891}{Neumann \& Brown, 2020}}

\end{frame}


\begin{frame}
  \frametitle{Example: 100\% renewable electricity system for Europe}

\begin{columns}[T]
  \begin{column}{7cm}
    \includegraphics[trim=0cm 0cm 0cm 0cm, clip, width=7cm]{graphics/space-00.pdf}
  \end{column}
  \begin{column}{6.5cm}

    Within 10\% of the optimum we can:
    \begin{itemize}
    \item Eliminate most grid expansion
    \item Exclude onshore or offshore wind or PV
    \item Exclude battery or most hydrogen storage
    \end{itemize}

    \vspace{.2cm}

    \alert{Robust conclusions}: wind, some transmission, some storage, preferably hydrogen storage, required for a cost-effective solution.

    \vspace{.2cm}

    This gives space to choose solutions with \alert{higher public acceptance}.
  \end{column}
\end{columns}


  \source{\href{https://arxiv.org/abs/1910.01891}{Neumann \& Brown, 2020}}

\end{frame}



\end{document}
