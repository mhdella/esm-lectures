% NB: use pdflatex to compile NOT pdftex.  Also make sure youngtab is
% there...

% converting eps graphics to pdf with ps2pdf generates way too much
% whitespace in the resulting pdf, so crop with pdfcrop
% cf. http://www.cora.nwra.com/~stockwel/rgspages/pdftips/pdftips.shtml




\documentclass[10pt,aspectratio=169,dvipsnames]{beamer}
\usetheme[color/block=transparent]{metropolis}

\usepackage[absolute,overlay]{textpos}
\usepackage{booktabs}
\usepackage[utf8]{inputenc}


\usepackage[scale=2]{ccicons}

\usepackage[official]{eurosym}


\usepackage{tikz}
\usetikzlibrary{arrows.meta}

\usepackage{xcolor}

\makeatletter
\def\mathcolor#1#{\@mathcolor{#1}}
\def\@mathcolor#1#2#3{%
  \protect\leavevmode
  \begingroup
    \color#1{#2}#3%
  \endgroup
}
\makeatother

%use this to add space between rows
\newcommand{\ra}[1]{\renewcommand{\arraystretch}{#1}}


\setbeamerfont{alerted text}{series=\bfseries}
\setbeamercolor{alerted text}{fg=Mahogany}
\setbeamercolor{background canvas}{bg=white}


\newcommand{\R}{\mathbb{R}}

\def\l{\lambda}
\def\m{\mu}
\def\d{\partial}
\def\cL{\mathcal{L}}
\def\co2{CO${}_2$}



% for sources http://tex.stackexchange.com/questions/48473/best-way-to-give-sources-of-images-used-in-a-beamer-presentation

\setbeamercolor{framesource}{fg=gray}
\setbeamerfont{framesource}{size=\tiny}


\newcommand{\source}[1]{\begin{textblock*}{5cm}(10.5cm,8.35cm)
    \begin{beamercolorbox}[ht=0.5cm,right]{framesource}
        \usebeamerfont{framesource}\usebeamercolor[fg]{framesource} Source: {#1}
    \end{beamercolorbox}
\end{textblock*}}

\usepackage{hyperref}


%\usepackage[pdftex]{graphicx}


\graphicspath{{graphics/}}

\DeclareGraphicsExtensions{.pdf,.jpeg,.png,.jpg}



\def\goat#1{{\scriptsize\color{green}{[#1]}}}


\newcommand{\ubar}[1]{\text{\b{$#1$}}}

\let\olditem\item
\renewcommand{\item}{%
\olditem\vspace{5pt}}

\title{Energy System Modelling\\ Summer Semester 2020, Lecture 16}
%\subtitle{---}
\author{
  {\bf Dr. Tom Brown}, \href{mailto:tom.brown@kit.edu}{tom.brown@kit.edu}, \url{https://nworbmot.org/}\\
  \emph{Karlsruhe Institute of Technology (KIT), Institute for Automation and Applied Informatics (IAI)}
}

\date{}


\titlegraphic{
  \vspace{0cm}
  \hspace{10cm}
    \includegraphics[trim=0 0cm 0 0cm,height=1.8cm,clip=true]{kit.png}

\vspace{5.1cm}

  {\footnotesize

  Unless otherwise stated, graphics and text are Copyright \copyright Tom Brown, 2020.
  Graphics and text for which no other attribution are given are licensed under a
  \href{https://creativecommons.org/licenses/by/4.0/}{Creative Commons
  Attribution 4.0 International Licence}. \ccby}
}

\begin{document}

\maketitle


\begin{frame}

  \frametitle{Table of Contents}
  \setbeamertemplate{section in toc}[sections numbered]
  \tableofcontents[hideallsubsections]
\end{frame}


\section{The World is Not a Perfect Optimization Model}


\begin{frame}
  \frametitle{We should be skeptical about models and modellers}
\begin{columns}[T]
  \begin{column}{7.5cm}

    \vspace{.2cm}
    \centering
    \includegraphics[width=7cm]{2019-01-10-IEEFA-EIA-coal-all-consumption-forecasts-470-x-395-v2-768x646}
  \end{column}
  \begin{column}{7.5cm}

    \vspace{.2cm}
    \centering
    \includegraphics[width=7cm]{auke.jpg}
  \end{column}
\end{columns}
\end{frame}

\begin{frame}
  \frametitle{We should be skeptical about models and modellers}

\begin{columns}[T]
  \begin{column}{9cm}

    \vspace{.2cm}
    \centering
    \includegraphics[width=9.5cm]{hubbert.png}
  \end{column}
  \begin{column}{5cm}

    \vspace{.2cm}
    \begin{itemize}
    \item Possible scenario projected from 1956 by US geologist M. King Hubbert
    \item Oil production in the US did indeed peak in the 1970s, but returned to peak height in last decade thanks to shale oil extraction with fracking
    \item Nuclear expanded but plateaued
      \item \alert{What might we be getting wrong in 2020?}
    \end{itemize}
  \end{column}
\end{columns}

\source{\href{http://www.energycrisis.com/Hubbert/1956/1956.pdf}{Hubbert, 1956}}
\end{frame}


\begin{frame}
  \frametitle{We should be skeptical about models and modellers}

  Models can:
  \vspace{-.3cm}
    \begin{itemize}
    \item \alert{under- or overestimate rates of change} (e.g. under: PV uptake, over: onshore wind in UK/Germany/Netherlands)
    \item \alert{underestimate social factors} (e.g. concern about nuclear / transmission / wind)
    \item \alert{extrapolate based on uncertain data} (e.g. oil reserves, learning curves for PV)
    \item \alert{focus on easy-to-solve rather than policy-relevant problems} (e.g. most research)
    \item \alert{neglect uncertainty} (e.g. in short-term due to weather forecasts, or in long-term due to cost, political uncertainty and technological development)
    \item \alert{neglect need for robustness} (e.g. securing energy system against contingencies, attack)
    \item \alert{neglect complex interactions of markets and incentive structures} (e.g. abuse of market power, non-linearities not represented in models, lumpiness, etc.)
    \item \alert{neglect non-linearities and non-convexities} (e.g. power flow, or also learning curves, behavioural effects, perverse local optima, many, many more)
    \end{itemize}

\end{frame}


\section{Robustness to Different Weather Years}


\begin{frame}
  \frametitle{Different Weather Years}

  Many of the simulations we looked at in this course, and many in the literature, used single weather years to determine optimal investments.

  This is problematic since:
  \begin{itemize}
  \item Weather changes from year to year
  \item There are decadal variations of wind
  \item Demand changes (particularly space heating demand during cold years)
  \end{itemize}

  But computing investments against 30 years of data (262,800 hours) is not feasible.

\end{frame}


\begin{frame}
  \frametitle{Different Weather Years}

  If we use different weather years to optimize sector-coupled European model with net-zero CO$_2$ emissions (including industry) we see broadly stable technology choices but variations in total system costs of up to 20\%.  NB: In real world cannot reoptimize investment every year!

  \centering
    \includegraphics[width=13cm]{lin-total_costs.png}

    \source{Lin Yang MA thesis}

\end{frame}


\begin{frame}
  \frametitle{Different Weather Years}

  Biggest changes are driven by space heating demand. Cold years (like 2010) are more expensive.

  \centering
    \includegraphics[width=14cm]{lin-heating.png}


    \source{Lin Yang MA thesis}

\end{frame}


\begin{frame}
  \frametitle{Different Weather Years}

  Optimal technology investments do not change dramatically from year to year. Here we show the mean capacities with standard deviation.

  \centering
    \includegraphics[width=14cm]{lin-generators.png}


    \source{Lin Yang MA thesis}

\end{frame}


\begin{frame}
  \frametitle{Different Weather Years}

  If we fix the optimal technology investments based on the weather of one year ($x$-axis), then run the dispatch over all 30 years (900 simulations in total), we can assess average curtailment and load-shedding. Using coldest year 2010 gives low load-shedding but high curtailment.

  \centering
    \includegraphics[width=14cm]{lin-run_30.png}

    \source{Lin Yang MA thesis}

\end{frame}


\begin{frame}
  \frametitle{Using 2010 investments}

  Using coldest year 2010 guarantees virtually no load-shedding in entire 30 years, but leads to excess energy in most years.
  Better to store excess energy from warmer years (e.g. chemically).

  \centering
    \includegraphics[width=14cm]{lin-use_2010.png}

    \source{Lin Yang MA thesis}

\end{frame}



\section{Effects of Climate Change on Energy System}


\begin{frame}
  \frametitle{Climate change}

  \begin{itemize}
  \item   What are the consequences of climate change for highly renewable energy systems?
  \item How will generation patterns for wind and solar change?
    \item What will be the effects on the dimensioning of wind,
      solar, storage, networks and backup generation?
  \end{itemize}

\end{frame}

\begin{frame}
  \frametitle{Climate change scenarios: RCP 8.5}

  Take a simulated dataset of how the weather would look
  between today and the year 2100 with a scenario of high
  concentrations of greenhouse gases.

  The scenario is called Representative Concentration Pathways 8.5
  (RCP~8.5), since it estimates a radiative forcing of $\Delta P = 8.5$~
  $\mathrm{W}/ \mathrm{m}^2$ (difference between insolation and energy
  radiated into space) at the end of the century. It is a \alert{worst-case scenario} and extrapolates current greenhouse gas emissions
  without reduction efforts (improbable given current trajectories of coal, renewables and EVs). This corresponds to a
  CO$_2$-equivalent-concentration (including all forcing agents) of
  approximately 1250 ppm (today around 410~pmm for CO$_2$) and an
  average temperature increase of $\Delta T = 3.7 \pm 1.1$~C at the
  end of the century, dependent on the model used.

  Compare historical values (HIS) to begin/middle/end of the century (B/M/EOC).

\end{frame}


\begin{frame}
  \frametitle{Changes to wind capacity factors}

  Left: historic (HIS) wind capacity factors 1970-2005

  Right: change at end of century (EOC) 2070-2100
  \begin{columns}[T]
    \begin{column}{9.5cm}

      \includegraphics[width=10cm]{cc-wind.png}
  \end{column}
  \begin{column}{4.5cm}
    \vspace{0.5cm}
    \begin{itemize}
    \item Small ($\sim 5\%$) average increase in Northern Europe
    \item Small ($\sim 5\%$) average decrease in Southern Europe
    \end{itemize}

  \end{column}
\end{columns}

      %corr_ensemble_sfcWind_abs.eps

  \source{\href{https://arxiv.org/abs/1805.11673}{Schlott et al, 2018}}

\end{frame}


\begin{frame}
  \frametitle{Changes to solar capacity factors}

  \begin{columns}[T]
  \begin{column}{9.5cm}
    \includegraphics[width=10cm]{cc-solar.png}
  \end{column}
  \begin{column}{4.5cm}
    \vspace{0.5cm}
    \begin{itemize}
    \item Small ($\sim 5\%$) increase in
      in Southern Europe around Mediterranean
    \item Smallish ($\sim 10\%$) decrease in Northern Europe (due to
      increased cloud cover)
    \item Solar results known to be a little unreliable because of
      cloud modelling etc.
    \end{itemize}

  \end{column}
\end{columns}


    %mean_ensemble_sfcWind_abs.eps
      %corr_ensemble_sfcWind_abs.eps


  \source{\href{https://arxiv.org/abs/1805.11673}{Schlott et al, 2018}}

\end{frame}

\begin{frame}
  \frametitle{Correlation Length}


  \begin{columns}[T]
  \begin{column}{5.5cm}
    \includegraphics[height=7.5cm]{simeon-correlation}
  \end{column}
  \begin{column}{8.5cm}
    The Pearson correlation coefficient of wind time series with a point in
        northern Germany decays exponentially with distance.
    Determine the \alert{correlation length} $L$ by fitting the
        function:
        \begin{equation*}
          \rho \sim e^{-\frac{x}{L}}
        \end{equation*}
        to the radial decay with distance $x$.

        \vspace{.2cm}
  \centering
    \includegraphics[height=4.5cm]{correlation_length.png}



  \end{column}
\end{columns}


  \source{\href{https://doi.org/10.1016/j.apenergy.2011.10.039}{Hagspiel et al, 2012}}
\end{frame}


\begin{frame}
  \frametitle{Changes to wind speed correlation lengths}

  \begin{columns}[T]
  \begin{column}{9.5cm}
    \includegraphics[width=10cm]{cc-correlation.png}
  \end{column}
  \begin{column}{5.3cm}
    \begin{itemize}
      \item Correlation lengths are longer in the North than
        the South because of big weather systems that roll in from
        the Atlantic to the North (in the South they get dissipated).
      \item With global warming, correlation lengths grow
        longer in the North and shorter in the South.
      \item This is because weather systems have more energy and are
        bigger in the North.
    \end{itemize}

  \end{column}
\end{columns}


  \source{\href{https://arxiv.org/abs/1805.11673}{Schlott et al, 2018}}

\end{frame}



\begin{frame}
  \frametitle{Effects of climate change on power system}

  Conclusions from study of effects on the power system:

    \begin{itemize}
    \item Most effects are small ($\sim 5-10\%$); total system costs increase by only 5\%.
     \item Longer correlation lengths see greater benefit from continental transmission.
     \item Impact of climate change is of a similar magnitude to the uncertainty between the different weather models.
       \item Not considered: Space heating and cooling demand changes may have bigger effect on overall energy system.
       \item Not considered: Impact of extreme weather events (storms, fires, droughts).
    \end{itemize}

    For more results, see `The Impact of Climate Change on a Cost-Optimal Highly Renewable European Electricity Network,' \url{https://arxiv.org/abs/1805.11673}

\end{frame}



%\section{Myopic Foresight for Weather Uncertainty}

%\section{Myopic Foresight for Multi-Decade Investment}

\section{Cost and Political Uncertainty}

\begin{frame}
  \frametitle{Power System Model: Sensitivity to Changing Solar Cost}

  In 30-node European electricity system with 95\% CO$_2$ reduction, change solar
  capital cost relative to default. NB: Even at zero solar cost, there
  is still wind. Why? Seasonality.\\
  LV 0: No cross-border grid, LV 125: compromise grid, LV Opt: optimal grid.

\centering
    \includegraphics[width=11cm]{sensitivity-solar_cost.png}

    \source{\href{https://arxiv.org/abs/1803.09711}{Schlachtberger et al, 2018}}
\end{frame}



\begin{frame}
  \frametitle{Power System Model: Sensitivity to Onshore Wind Installable Potential}

  In electricity system with 95\% CO$_2$ reduction, reduce installable
  potential for onshore wind. Onshore substituted with offshore at only small
  extra system cost. BUT assumes sufficient grid capacity within each country to get offshore from coast to load.

\centering
    \includegraphics[width=11cm]{sensitivity-onshore_wind.png}

    \source{\href{https://arxiv.org/abs/1803.09711}{Schlachtberger et al, 2018}}
\end{frame}


\begin{frame}
  \frametitle{Sensitivity of Optimisation to Cost, Weather Data and Policy Constraints}

  See Schlachtberger et al, `Cost optimal scenarios of a future highly renewable European
  electricity system: Exploring the influence of weather data, cost
  parameters and policy constraints,' 2018,
  \url{https://arxiv.org/abs/1803.09711}

\end{frame}

\section{Near-Optimal Energy Systems}

\begin{frame}
  \frametitle{Flat directions near optimum}

  Both for changing transmission expansion AND onshore wind installable potentials, we've seen that total system costs are \alert{flat around the optimum}.

  Can we explore this \alert{near-optimal space} more systematically?

  \vspace{.5cm}
\begin{columns}[T]

  \begin{column}{6cm}
  \includegraphics[width=6cm]{costs_per_cluster_181}
  \end{column}
  \begin{column}{8cm}
    \includegraphics[width=8cm]{sensitivity-onshore_wind.png}
  \end{column}
\end{columns}

\end{frame}
% optimal
\begin{frame}
  \frametitle{Large Space of Near-Optimal Energy Systems}

  There is a \alert{large degeneracy} of different possible energy systems close to the optimum.

  \vspace{.2cm}
  \centering
  \includegraphics[page=2, trim=0.3cm 16.02cm 0cm 3.05cm, clip, height=0.83\textheight]{graphics/sketch.pdf}
  \source{Fabian Neumann}
\end{frame}

% epsilon constraint
\begin{frame}
  \frametitle{Large Space of Near-Optimal Energy Systems}

  Consider the part of the feasible space within $\varepsilon$ of the optimum $f(x^*)$.

  \vspace{.2cm}
  \centering
  \includegraphics[page=1, trim=0.3cm 0.7cm 0cm 18.35cm, clip, height=0.83\textheight]{graphics/sketch.pdf}
  \source{Fabian Neumann}
\end{frame}

% mga
\begin{frame}
  \frametitle{Large Space of Near-Optimal Energy Systems}

  Now within $\varepsilon$ of the optimum $f(x^*)$, try minimising or maximising $x$, to probe space.

  \vspace{.2cm}
  \centering
  \includegraphics[page=1, trim=0.3cm 15.45cm 0cm 3.6cm, clip, height=0.83\textheight]{graphics/sketch.pdf}
  \source{Fabian Neumann}
\end{frame}



% 2-dimensional
\begin{frame}
  \frametitle{Large Space of Near-Optimal Energy Systems}

  NB: Decision space of variables is multi-dimensional, so can probe only one direction at a time.

  \vspace{.2cm}
  \centering
  \includegraphics[page=1, trim=8.8cm 11.5cm 9.4cm 14.8cm, clip, height=0.83\textheight]{graphics/sketch.pdf}
  \source{Fabian Neumann}
\end{frame}



\begin{frame}{Application: Highly-Renewable European Electricity System}

  Apply this technique to a 100-node model of the European electricity with 100\% renewable energy.
  \begin{enumerate}
    \item Find the \alert{least-cost power system}.
    ~\\
    \item For $\varepsilon\in\{0.5,1,\dots,10\} \%$ \alert{minimise/maximise} investment in
    \begin{itemize}
      \item generation capacity {(onshore and/or offshore wind, solar)},
      \item storage capacity {(hydrogen, batteries, total storage)} and
      \item transmission volume {(HVAC lines and HVDC links)}
    \end{itemize}
    such that \alert{total annual system costs increase by less than $\varepsilon$}.
  \end{enumerate}

  \vspace{.5cm}
  Methodology adapted from Method to Generate Alternatives (MGA) but `alternatives' are forced in politically-interesting directions.

  \source{Fabian Neumann}

\end{frame}



\begin{frame}
  \frametitle{Example: 100\% renewable electricity system for Europe}

\begin{columns}[T]
  \begin{column}{7cm}
    Capacity expansion in optimum:

    \includegraphics[width=7cm]{no-optimum.png}
  \end{column}
  \begin{column}{7cm}

    $\varepsilon =$ 10\% above optimum, minimise new grid:

    \includegraphics[width=7cm]{no-trans-10.png}
  \end{column}
\end{columns}



  \source{\href{https://arxiv.org/abs/1910.01891}{Neumann \& Brown, 2020}}

\end{frame}


\begin{frame}
  \frametitle{Example: 100\% renewable electricity system for Europe}

\begin{columns}[T]
  \begin{column}{7cm}
    \includegraphics[trim=0cm 0cm 0cm 0cm, clip, width=7cm]{graphics/space-00.pdf}
  \end{column}
  \begin{column}{6.5cm}

    Within 10\% of the optimum we can:
    \begin{itemize}
    \item Eliminate most grid expansion
    \item Exclude onshore or offshore wind or PV
    \item Exclude battery or most hydrogen storage
    \end{itemize}

    \vspace{.2cm}

    \alert{Robust conclusions}: wind, some transmission, some storage, preferably hydrogen storage, required for a cost-effective solution.

    \vspace{.2cm}

    This gives space to choose solutions with \alert{higher public acceptance}.
  \end{column}
\end{columns}


  \source{\href{https://arxiv.org/abs/1910.01891}{Neumann \& Brown, 2020}}

\end{frame}



\begin{frame}[fragile]
  \frametitle{Flat directions allow society to choose based on other criteria}



\begin{columns}[T]

  \begin{column}{5cm}


\centering
\includegraphics[width=5cm]{nein_zur_monstertrasse}

  \end{column}


  \begin{column}{6.7cm}

    \vspace{.5cm}

    This flatness may allow us to choose solutions with \alert{higher
      public acceptance} at only \alert{small extra cost}.

    \vspace{.5cm}

    These trade-offs will occupy us for the next 30 years!

    \vspace{.5cm}

\includegraphics[width=7cm]{Protestplakat-gegen-den-Bau-von-Windraedern-in-Hamburg-Deutschland.jpg}

  \end{column}

\end{columns}

\end{frame}


\begin{frame}{\textbf{Dependencies:} Extremes cannot be achieved simultaneously}

  \begin{columns}

    \begin{column}{0.4\textwidth}
      \begin{center}
        \vspace{-0.5cm}
        \footnotesize Optimal System Layout\\
        \includegraphics[height=0.8\textheight]{graphics/capacity-bar.pdf}
      \end{center}
    \end{column}
    %\vrule
    \begin{column}{0.6\textwidth}
      \begin{center}
        \vspace{.3cm}
        \includegraphics[height=0.9\textheight]{graphics/capacity-bars.pdf}
      \end{center}
    \end{column}

  \end{columns}

\end{frame}

\begin{frame}
  \frametitle{Near-Optimal Systems: Conclusions}

  \begin{itemize}
  \item Optimizing a single model gives a \alert{false sense of exactness}.
  \item There are many uncertainties about cost assumptions and political targets.
  \item There are also \alert{structural model uncertainties} since the feasible space can be very \alert{flat} near the optimum, such that the solution chosen is random within flat area.
  \item We can use these techniques to probe the \alert{near-optimal space}.
  \item This gives us fuzzier but \alert{more robust} conclusions (e.g. need wind, some transmission and some long-term storage for a cost-effective solution).
    \item It also allows us to find cost-effective solutions with \alert{higher public acceptance}.

  \end{itemize}

  More details: Fabian Neumann, Tom Brown, ``The Near-Optimal Feasible Space of a Renewable Power System Model,'' 2020, accepted to PSCC 2020, \url{https://arxiv.org/abs/1910.01891}.

\end{frame}
\end{document}
