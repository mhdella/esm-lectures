% NB: use pdflatex to compile NOT pdftex.  Also make sure youngtab is
% there...

% converting eps graphics to pdf with ps2pdf generates way too much
% whitespace in the resulting pdf, so crop with pdfcrop
% cf. http://www.cora.nwra.com/~stockwel/rgspages/pdftips/pdftips.shtml




\documentclass[10pt,aspectratio=169,dvipsnames]{beamer}
\usetheme[color/block=transparent]{metropolis}

\usepackage[absolute,overlay]{textpos}
\usepackage{booktabs}
\usepackage[utf8]{inputenc}

\usepackage{tikz}
\usetikzlibrary{arrows.meta}


\usepackage[scale=2]{ccicons}

\usepackage[official]{eurosym}


%use this to add space between rows
\newcommand{\ra}[1]{\renewcommand{\arraystretch}{#1}}


\newcommand{\R}{\mathbb{R}}



\setbeamerfont{alerted text}{series=\bfseries}
\setbeamercolor{alerted text}{fg=Mahogany}
\setbeamercolor{background canvas}{bg=white}


\def\l{\lambda}
\def\m{\mu}
\def\d{\partial}
\def\cL{\mathcal{L}}
\def\co2{CO${}_2$}
\def\bra#1{\left\langle #1\right|}
\def\ket#1{\left| #1\right\rangle}
\newcommand{\braket}[2]{\langle #1 | #2 \rangle}
\newcommand{\norm}[1]{\left\| #1 \right\|}
\def\corr#1{\Big\langle #1 \Big\rangle}
\def\corrs#1{\langle #1 \rangle}



% for sources http://tex.stackexchange.com/questions/48473/best-way-to-give-sources-of-images-used-in-a-beamer-presentation

\setbeamercolor{framesource}{fg=gray}
\setbeamerfont{framesource}{size=\tiny}


\newcommand{\source}[1]{\begin{textblock*}{5cm}(10.5cm,8.35cm)
    \begin{beamercolorbox}[ht=0.5cm,right]{framesource}
        \usebeamerfont{framesource}\usebeamercolor[fg]{framesource} Source: {#1}
    \end{beamercolorbox}
\end{textblock*}}

\usepackage{hyperref}

\usepackage{tikz}


\usepackage{circuitikz}

%\usepackage[pdftex]{graphicx}


\graphicspath{{graphics/}}

\DeclareGraphicsExtensions{.pdf,.jpeg,.png,.jpg,.gif}



\def\goat#1{{\scriptsize\color{green}{[#1]}}}



\let\olditem\item
\renewcommand{\item}{%
\olditem\vspace{5pt}}

\title{Energy System Modelling\\ Summer Semester 2020, Lecture 5}
%\subtitle{---}
\author{
  {\bf Dr. Tom Brown}, \href{mailto:tom.brown@kit.edu}{tom.brown@kit.edu}, \url{https://nworbmot.org/}\\
  \emph{Karlsruhe Institute of Technology (KIT), Institute for Automation and Applied Informatics (IAI)}
}

\date{}

\titlegraphic{
  \vspace{0cm}
  \hspace{10cm}
    \includegraphics[trim=0 0cm 0 0cm,height=1.8cm,clip=true]{kit.png}

\vspace{5.1cm}

  {\footnotesize

  Unless otherwise stated, graphics and text are Copyright \copyright Tom Brown, 2020.
  Graphics and text for which no other attribution are given are licensed under a
  \href{https://creativecommons.org/licenses/by/4.0/}{Creative Commons
  Attribution 4.0 International Licence}. \ccby}
}

\begin{document}

\maketitle


\begin{frame}

  \frametitle{Table of Contents}
  \setbeamertemplate{section in toc}[sections numbered]
  \tableofcontents[hideallsubsections]
\end{frame}


\section{Principles of electricity storage}



\begin{frame}
 \frametitle{Recall from Previous Lectures}
 Conceptual options to balance wind and solar (avoiding need for backup and curtailment):
 \begin{itemize}
  \item Transmission grid (last lecture)
  \item \alert{Storage}
  \item Demand-side management
  \item Sector coupling
 \end{itemize}

\end{frame}

\begin{frame}
  \frametitle{Basic idea of storage}

  Networks were used to shift power imbalances between different
  places, i.e. in \alert{space}. Electricity storage can shift power in \alert{time}.

  \vspace{0.5cm}
  \centering
  \begin{tikzpicture}
\node[anchor=south west,inner sep=0] (image) at (0,0) {\includegraphics[width=11cm]{mismatch-2011-07-01-2011-07-07}};
\draw[red,line width=5,->] (4,3) -- (4,4) -- (6.5,4) -- (6.5,2.5);
  \end{tikzpicture}

\end{frame}



\begin{frame}
  \frametitle{Storage: Power Versus Energy Capacity}

  For a storage unit, we have to distinguish between the \alert{power capacity} (MW) at which we can discharge (dispatch) or charge the storage, and the \alert{energy capacity} (MWh) it can store.

  \begin{columns}[T]
    \begin{column}{7cm}

  Examples:
  \begin{itemize}
  \item A Tesla battery electric vehicle can charge with a power of 11~kW at home or 100-150~kW at a supercharger. The Model S has an energy capacity of 100 kWh.
    \item The Hornsdale utility-scale battery in South Australia has a power capacity of 100~MW and an energy capacity of 185~MWh.
  \end{itemize}

    \end{column}
    \begin{column}{6cm}

          \includegraphics[trim=0 0cm 0 0cm,width=5cm,clip=true]{tesla-charging.jpg}

\includegraphics[trim=0 0cm 0 0cm,width=5cm,clip=true]{tesla-hornsdale-powerpack-1_33945.jpg}
    \end{column}
  \end{columns}
\end{frame}

\begin{frame}
  \frametitle{Storage consistency}

  Storage units, such as batteries or hydrogen storage, labelled by $r$, can both
  dispatch/discharge power within its discharging capacity (in MW):
  \begin{equation*}
    0 \leq g_{i,r,t,\textrm{discharge}} \leq G_{i,r,\textrm{discharge}}
  \end{equation*}
  and consume power to charge the storage within its charging capacity (in MW):
  \begin{equation*}
    0 \leq g_{i,r,t,\textrm{charge}} \leq G_{i,r,\textrm{charge}}
  \end{equation*}
  The total power (positive when discharging, negative when charging) can then be written:
  \begin{equation*}
    g_{i,r,t} = g_{i,r,t,\textrm{discharge}}  - g_{i,r,t,\textrm{charge}}
  \end{equation*}
  There is also a limit on the total energy $e_{i,r,t}$ at each time:
  \begin{equation*}
    0 \leq e_{i,r,t} = e_{i,r,0} -\sum_{t'=1}^{t} g_{i,r,t'} \leq E_{i,r}
  \end{equation*}
  where $E_{i,r}$ is the energy capacity (in MWh). Or in iterative form
  \begin{align*}
    0 \leq e_{i,r,t} = e_{i,r,t-1} + g_{i,r,t,\textrm{charge}} -  g_{i,r,t,\textrm{discharge}} \leq E_{i,r}
  \end{align*}

\end{frame}

\begin{frame}
  \frametitle{Incorporation in power balance with generation, demand and network}

  We can then incorporate the storage power $g_{i,r,t}$ in our power imbalance for each node $i$ and each time $t$ from last lecture:
  \begin{equation*}
    p_{i,t} = \sum_{s} g_{i,s,t} + \sum_{r} g_{i,r,t} - d_{i,t} = \sum_{\ell} K_{i\ell} f_{\ell,t}
  \end{equation*}
  ($s$ runs over generation technologies, $r$ over storage technologies, $\ell$ over network lines)

  The nodes are linked in space by the network and in time by the consistency for the storage energy.

  If we expand the storage power $g_{i,r,t}$ into its charging and discharging parts:
  \begin{equation*}
    p_{i,t} = \sum_{s} g_{i,s,t} + \sum_{r}\left(g_{i,r,t,\textrm{discharge}}  - g_{i,r,t,\textrm{charge}}\right) - d_{i,t} = \sum_{\ell} K_{i\ell} f_{\ell,t}
  \end{equation*}
  we see that the discharging part has the sign of a generator putting power into the system, while the charging part acts like a demand extracting power from the system.

\end{frame}
\begin{frame}
  \frametitle{Storage consistency: continuous version}

  In the previous slide we had discrete time points $t\in\{0,1,2,\dots\}$.

  The discrete equation for the total energy (sometimes called the \alert{state of charge}):
    \begin{equation*}
    0 \leq e_{i,r,t} = e_{i,r,0} -\sum_{t'=1}^{t} g_{i,r,t'} \leq E_{i,r}
  \end{equation*}
    is nothing other than an integration over time $t$.

    So if we write $g_{i,r,t}$ and $e_{i,r,t}$ as continuous functions of $t$, $g_{i,r}(t)$ and $e_{i,r}(t)$, we get an integration:
    \begin{equation*}
       0 \leq e_{i,r}(t) = e_{i,r}(0) - \int_0^t g_{i,r}(t') dt'  \leq E_{i,r}
    \end{equation*}

\end{frame}


\begin{frame}
  \frametitle{Continuous example}

  Consider a single node (i.e. no network) with a constant demand
  \begin{equation*}
    d(t) = D
  \end{equation*}
  and a renewable wind generator with a capacity $G = 2D$ and an
  availability time series
  \begin{equation*}
    G(t) = \frac{1}{2} \left(1 + \sin\left(\frac{2\pi}{T} t\right)\right)
  \end{equation*}
  so that it oscillates with period $T$ and on average produces enough energy for the demand
  \begin{equation*}
    \langle G(t) G \rangle = D
  \end{equation*}

  \alert{Question}: What are the power and energy capacities of the ideal storage unit to balance this system?
\end{frame}



\begin{frame}
  \frametitle{Continuous example}

  For $D =1, T = 2\pi$:

  \centering
  \includegraphics[width=14cm]{demand-wind}
\end{frame}



\begin{frame}
  \frametitle{Residual demand}

  Our residual demand or mismatch is now
  \begin{equation*}
    m(t) = d(t) - G G(t) = D - D \left(1 + \sin\left(\frac{2\pi}{T} t\right)\right) =  -D\sin\left(\frac{2\pi}{T} t\right)
  \end{equation*}

  For $D =1, T = 2\pi$:

  \centering
\includegraphics[width=11cm]{mismatch-sin}


\end{frame}

\begin{frame}
  \frametitle{Storage Power}

  To balance this, we need a storage unit with power profile $g_r(t)$ such that the node balances:
  \begin{equation*}
    0 = p(t) =  GG(t) + g_r(t) - d(t) = g_r(t) - m(t)
  \end{equation*}
  i.e.
  \begin{equation*}
    g_r(t) =  d(t) - G G(t) = m(t) =   -D\sin\left(\frac{2\pi}{T} t\right)
  \end{equation*}
  This will have power capacities $G_{r,\textrm{discharge}} = G_{r,\textrm{charge}} = D$.   For $D =1, T = 2\pi$:


  \centering
  \includegraphics[width=9cm]{storage-power}

\end{frame}


\begin{frame}
  \frametitle{Storage Energy}

  How much energy capacity $E_{r}$ do we need? The energy profile is:
  \begin{equation*}
    e_r(t) = \int_{0}^{t} (-g_r(t')) dt' = D \int_0^t \sin\left(\frac{2\pi}{T} t'\right) dt' =  \frac{TD}{2\pi}\left[ 1 - \cos\left(\frac{2\pi}{T} t\right)\right]
  \end{equation*}
  This peaks at $t=\frac{T}{2}$  so $E_r = \max_t\{e_r(t)\} = \frac{TD}{\pi}$. Faster oscillations, i.e. shorter periods, $\Rightarrow$ less energy capacity. So for $D =1, T = 2\pi$, maximum is $E_r = 2$:

  \centering
  \includegraphics[width=10cm]{storage-energy}

\end{frame}



\begin{frame}
  \frametitle{Fourier coefficients of wind and solar in Germany}

  Although wind and solar are not perfect sine waves, if we decompose
  the time series in Fourier components, we do see \alert{dominant
    frequencies} which can help us understand how to dimension
  storage.

  \vspace{1cm}

  \includegraphics[width=6.9cm]{DE-solar-spectrum}
  \includegraphics[width=6.9cm]{DE-onwind-spectrum}


\end{frame}


\begin{frame}
  \frametitle{Storage Energy: concrete examples}

  How does our formula $E_r = \frac{TD}{\pi}$ look for different
  generation technologies with simplified sinusoidal profiles?

  Consider a simplified demand of $D = 1$ MW.
  \ra{1.05}
  \begin{table}[!t]
    \begin{tabular}{lllrr}
      \toprule
      quantity & symbol & units & solar & wind\\
      \midrule
      generation capacity &$G$ & MW & 2 & 2 \\
      storage power capacity & $G_r$ & MW & 1 & 1 \\
      period & $T$ & h & 24 & $7 \cdot 24 = 168$\\
      storage energy capacity & $E_r$ & MWh & 7.6 &  53 \\
      \bottomrule
    \end{tabular}
  \end{table}
  Faster daily oscillations of solar need smaller storage capacity than weekly oscillations of wind.

  NB: In reality of course solar and wind are not perfect sine waves...
\end{frame}


\begin{frame}
  \frametitle{Efficiency and losses}

  There are a few extra details to add now. First, no renewable has a perfectly regular sinusoidal profile.

  Second, the iterative integration equation for the storage energy
  \begin{align*}
    e_{i,r,t} = e_{i,r,t-1} + g_{i,r,t,\textrm{store}} -  g_{i,r,t,\textrm{dispatch}}
  \end{align*}
  needs to be amended for \alert{efficiencies} $\eta\in [0,1]$ (corresponding to \alert{losses} $1-\eta$)
  \begin{align*}
    e_{i,r,t} = \eta_0e_{i,r,t-1} + \eta_1g_{i,r,t,\textrm{store}} -  \eta_2^{-1} g_{i,r,t,\textrm{dispatch}}
  \end{align*}
  $1-\eta_0$ corresponds to \alert{standing losses} or \alert{self-discharge}, $\eta_1$ to the \alert{charging efficiency} and $\eta_2$ to the \alert{discharging efficiency}.

\end{frame}


\begin{frame}
  \frametitle{Different storage units have different parameters}

  We can relate the power capacity $G_r$ to the energy capacity $E_r$
  with the maximum number of hours the storage unit can be charged at full
power before the energy capacity is full, $E_r =
  \textrm{max-hours}*G_r$.

  \ra{1.05}
  \begin{table}[!t]
    \begin{tabular}{lrrrr}
      \toprule
      & Battery & Hydrogen & Pumped-Hydro & Water Tank\\
      \midrule
      $\eta_0$ & $1-\varepsilon$ & $1-\varepsilon$ & $1-\varepsilon$ & depends on size  \\
      $\eta_1$ & 0.9 & 0.75 & 0.9 & 0.9 \\
      $\eta_2$ & 0.9 & 0.58 & 0.9 & 0.9 \\
      max-hours & 2-10 & weeks & 4-10 & depends on size \\
      euro per kW [$G_r$] &300 &500+450 & depends& low \\
      euro per kWh [$E_r$] &200 & 10 &depends&low \\
      \bottomrule
    \end{tabular}
  \end{table}
Parameters are roughly based on
\href{http://www.sciencedirect.com/science/article/pii/S0378775312014759}{Budischak
  et al, 2012} with projections for 2030.


\end{frame}

\begin{frame}
  \frametitle{Different storage units have different use cases}

  Consider the cost of a storage unit with 1~kW of power capacity, and different energy capacities.

  The total losses are given by the \alert{round-trip losses} in and
  out of the storage $1- \eta_1\cdot \eta_2$.

  \ra{1.05}
  \begin{table}[!t]
    \begin{tabular}{lrr}
      \toprule
      & Battery & Hydrogen \\
      \midrule
      losses & 1 - 0.9$^2$ = 0.19 & 1 - 0.58*0.75 = 0.57 \\
      \euro{} for 2 kWh & 300 + 2x200 = 700 & 950 + 2x10 = 970 \\
      \euro{} for 100 kWh & 300 + 100x200 = 20300 & 950 + 100x10 = 1950 \\
      \bottomrule
    \end{tabular}
  \end{table}

  Battery has lower losses and is cheaper for short storage periods.

  Hydrogen has higher losses but is much cheaper for long storage periods (e.g. several days).


  \alert{You try}: Explore use cases using \url{https://model.energy} for Mexico (solar+battery) and Ireland (wind+hydrogen).
\end{frame}


\section{Power-to-Gas}

\begin{frame}
  \frametitle{Power-to-Gas (P2G)}

  \begin{columns}[T]
    \begin{column}{4cm}
      \includegraphics[trim=0 0cm 0 0cm,width=5cm,clip=true]{pem_electrolyzer.png}


      \vspace{.4cm}

      \includegraphics[trim=0 0cm 0 0cm,width=5cm,clip=true]{Methanation_of_CO2_circle.png}
    \end{column}
    \begin{column}{7cm}

      Power-to-Gas/Liquid (P2G/L) describes concepts to use electricity to
      electrolyse water to \alert{hydrogen} H$_2$ (and oxygen O$_2$).
      We can combine hydrogen with carbon oxides to get
      \alert{hydrocarbons} such as methane CH$_4$ (main component of
      natural gas) or liquid fuels C$_n$H$_m$.

      Used for \alert{hard-to-defossilise sectors}:
            \begin{itemize}
            \item
              \alert{dense fuels} for transport (planes, ships)
            \item \alert{steel-making} \& \alert{chemicals industry}
              \item \alert{high-temperature heat} or \alert{heat for buildings}
                \item \alert{backup energy} for cold low-wind winter periods, i.e. as storage
            \end{itemize}
    \end{column}
  \end{columns}

\end{frame}
\begin{frame}
  \frametitle{Power to Transport Fuels}

    \begin{columns}[T]
    \begin{column}{7cm}
      \includegraphics[width=7cm]{fuel_density-science.png}
    \end{column}
    \begin{column}{7cm}
      \begin{itemize}
      \item  Hydrogen has a very good gravimetric density (MJ/kg) but poor volumetric density (MJ/L).
      \item      Liquid hydrocarbons provide much better volumetric density for e.g. aviation.
      \item    WARNING: This graphic shows the thermal content of the fuel, but the \alert{conversion efficiency} of e.g. an electric motor for battery electric or fuel cell vehicle is much better than an internal combustion engine.
      \end{itemize}
      \end{column}
    \end{columns}
    \source{\href{https://doi.org/10.1126/science.aas9793}{Davis et al, 2018}}
\end{frame}

\begin{frame}
 \frametitle{Power to Gas Concept as Storage}

 \centering
 \includegraphics[width=12cm]{solar-hydrogen-system-illustration.jpg}

     \source{Buildipedia}
\end{frame}


\begin{frame}
  \frametitle{Power-to-Gas (P2G)}

  \begin{columns}[T]
    \begin{column}{5cm}
      \includegraphics[trim=0 0cm 0 0cm,width=4.5cm,clip=true]{salt_caverns.jpg}

      \vspace{.4cm}

      \includegraphics[trim=0 0cm 0 0cm,width=4.5cm,clip=true]{Gas-pipeline.jpg}
    \end{column}

    \begin{column}{7cm}
            \begin{itemize}
      \item  Gases and liquids are easy to \alert{store} and \alert{transport} than electricity.
        \item Storage capacity of the German natural gas network in terms of energy: ca 230 TWh. Europe wide it is 1100 TWh (see \href{https://agsi.gie.eu/}{\bf\color{blue}\underline{online table}}). In addition, losses in the gas network are small.
\item   (NB: Volumetric energy density of hydrogen, i.e. MWh/m$^3$, is around three times lower than natural gas.)
\item      Pipelines can carry many GW underground, out of sight.
            \end{itemize}
    \end{column}

\end{columns}

\end{frame}




\begin{frame}
 \frametitle{German Natural Gas Grid}
   \begin{figure}
     \includegraphics[width=6cm]{gasgrid.jpg}
  \caption{Buildipedia}
  \end{figure}
\end{frame}


\begin{frame}
  \frametitle{German Gas Transmission Network Operators Plan a Hydrogen Network}
    \begin{columns}[T]
    \begin{column}{7cm}

      \includegraphics[width=7cm]{fnb_gas_h2_startnetz_2030.jpg}
    \end{column}
    \begin{column}{6cm}
      \begin{itemize}
      \item       German Gas Transmission Network Operators have published a plan for a new nationwide hydrogen network to take hydrogen between sites of production (e.g. electrolysis near the coast where offshore wind is connected), sites of storage (underground caverns) and consumers of hydrogen (industry, etc.).
      \item 90\% of planned 2050 hydrogen network converts old natural gas pipelines; only 10\% needs to be built new.
      \end{itemize}
    \end{column}
    \end{columns}

    \source{\href{https://www.fnb-gas.de/fnb-gas/veroeffentlichungen/pressemitteilungen/fernleitungsnetzbetreiber-veroeffentlichen-h2-startnetz-2030/}{Fernleitungsnetzbetreiber}}
\end{frame}



\begin{frame}
  \frametitle{Electrolysis}

  \centering
     \includegraphics[width=10cm]{electrol.jpg}

     \source{Hyperphysics, Georgia State University}


\end{frame}


\begin{frame}
 \frametitle{Thermodynamic Calculation Electrolysis}
  \begin{equation*}
  H_2 O \rightarrow H_2 + \frac{1}{2} O_2
 \end{equation*}

 For one mole at conditions 298 K and one atmosspheric pressure

 \begin{table}
  \begin{center}
\begin{tabular}{llll}
x & $H_2$ & $O_2$ & $H_2 O$\\
Entropy [J/K] & 130.7 & 205.1 & 69.9\\
Enthalpy [kJ] & 0 & 0 & -285.8
  \end{tabular}
  \end{center}

 \end{table}

 Gibbs free energy $dG = dH - TdS$,
 \begin{align*}
 \Delta G = 285.8 kJ - 48.7 kJ = 237.1 kJ
 \end{align*}

\end{frame}


\begin{frame}
  \frametitle{Fuel Cell}

  \centering
     \includegraphics[width=8cm]{fuelcell.jpg}

     \source{Hyperphysics, Georgia State University}

\end{frame}

\begin{frame}
 \frametitle{Thermodynamics of Fuel Cell}

  Again: one mole at conditions 298 K and one atmosspheric pressure

 \begin{equation*}
  H_2 O \rightarrow H_2 + \frac{1}{2} O_2
 \end{equation*}



 Gibbs free energy $dG = dH - TdS$,
 \begin{align*}
 \Delta G = 285.8 kJ - 48.7 kJ = 237.1 kJ
 \end{align*}

 max theoretical efficiency
 \begin{align*}
   \frac{\Delta G}{\Delta U} = 0.83
  \end{align*}

\end{frame}



\section{Demand-Side Management (DSM)}


\begin{frame}
 \frametitle{Recall from Previous Lectures}
 Conceptual options to balance the power system:
 \begin{itemize}
  \item Transmission grid
  \item Storage
  \item \alert{Demand-side management}
  \item Sector coupling
 \end{itemize}

\end{frame}

\begin{frame}
  \frametitle{Basic Idea of Storage}

 Basic idea of storage: move supply:
  \vspace{0.5cm}

  \centering
  \begin{tikzpicture}
\node[anchor=south west,inner sep=0] (image) at (0,0) {\includegraphics[width=11cm]{mismatch-2011-07-01-2011-07-07}};
\draw[red,line width=5,->] (4,3) -- (4,4) -- (6.5,4) -- (6.5,2.5);
  \end{tikzpicture}
\end{frame}

\begin{frame}
  \frametitle{Basic Idea of Demand-Side Management}

 Modify demand instead of generation!
  \vspace{0.5cm}

  \centering
  \begin{tikzpicture}
\node[anchor=south west,inner sep=0] (image) at (0,0) {\includegraphics[width=11cm]{mismatch-2011-07-01-2011-07-07}};
\draw[red,line width=5,<-] (4,3) -- (4,4) -- (6.5,4) -- (6.5,2.5);
  \end{tikzpicture}

\end{frame}

\begin{frame}
  \frametitle{Demand-Side Management / Demand-Side Response}

  Modification of the Demand for energy through various means such as price incentives

  Charge consumers based on the true price of utilities at the time of consumption

  Issues: higher utility cost for consumers, communication infrastructure, synchronisation, cost, privacy, hacking

\end{frame}

\begin{frame}
 \frametitle{Cases of DSM: Efficiency Measures}
 \begin{minipage}[t]{0.5\textwidth}
    \begin{figure}
  \includegraphics[width=1.\textwidth]{efficiency-crop.pdf}
  \end{figure}
\end{minipage}\hfill
\begin{minipage}[t]{0.5\textwidth}
\begin{itemize}
 \item Permanent reduction of the demand by use of more efficient appliances
 \begin{itemize}
 \item washing machines
 \item refrigerators
 \item water heaters
 \end{itemize}
 \item Germany: Reduction of 25\% of gross electrical energy by 2050 compared to 2008
\end{itemize}

\end{minipage}


\end{frame}

\begin{frame}
 \frametitle{Cases of DSM: Peak Shaving}

  \begin{minipage}[t]{0.5\textwidth}
     \begin{figure}
  \includegraphics[width=1.\textwidth]{peakshaving-crop.pdf}
  \end{figure}
\end{minipage}\hfill
\begin{minipage}[t]{0.5\textwidth}
\begin{itemize}
\item Infrastructure designed for peak demand situations
\item Commercial consumers often charged based on their peak demand
\end{itemize}
\end{minipage}

\end{frame}

\begin{frame}
 \frametitle{Cases of DSM: Load Shifting}

   \begin{minipage}[t]{0.5\textwidth}
     \begin{figure}
  \includegraphics[width=1.\textwidth]{loadshifting-crop.pdf}
  \end{figure}
\end{minipage}\hfill
\begin{minipage}[t]{0.5\textwidth}
\begin{itemize}
\item Shift electrical demand from times of deficits to times of surpluses
\item  provide price incentives to cause load shifting via smart meters
\item different price incentive schemes possible, e.g., time of use prices, seasonal prices, etc.
\end{itemize}
\end{minipage}

\end{frame}
\begin{frame}
 \frametitle{Modelling Approach for DSM}
  \begin{itemize}
  \item loads into different categories with assumed max. shifting periods (e.g., 8 hours for household applications)
  \item shifting charges a virtual storage buffer
  \begin{align}
 P_n[R_n(t)](t) &= R_n(t) - L_n(t).
\end{align}
\item filling level is consequently given by
\begin{align}
 E_n[R_n(t)](t) &= \int_0^t P_n[R_n(t')](t') dt'
\end{align}
\item constraints by shifting periods, e.g.,
\begin{align}
  E^{+}_n(t) &= \int_t^{t+\Delta t} L_n(t') dt'
\end{align}
 \end{itemize}

\end{frame}
\begin{frame}
\frametitle{Modelling Approach for DSM}
\centering
\includegraphics[width=12cm]{timeseries-crop.pdf}
 % timeseries-crop.pdf: 0x0 pixel, 0dpi, 0.00x0.00 cm, bb=
\end{frame}

\begin{frame}
\frametitle{Modelling Approach for DSM}
Load shifting supports system integration of variable renewables, especially PV

% backup_vs_shareofrenewables-crop.pdf: 0x0 pixel, 300dpi, 0.00x0.00 cm, bb=
\centering
\includegraphics[width=0.75\textwidth]{optimal_mix-crop.pdf}
 % optimal_mix-crop.pdf: 0x0 pixel, 300dpi, 0.00x0.00 cm, bb=
 \source{Kies et al., Energies, 2016}
 \end{frame}
  \begin{frame}
   \frametitle{Demand-Side Management (DSM) Summary}

   \begin{itemize}
   \item Demand-side management can contribute to successful power system operation
     \item Efficiency first!
    \item ``Daily'' scale supports PV integration
    \item Building infrastructure for DSM can be cost-intensive and cause minor additional energy consumption
      \item Needs a careful consideration of constraints from consumer side
    \item Synchronisation via pricing can amplify fluctuations
      \item Other concerns: hacking and privacy
   \end{itemize}


  \end{frame}






\end{document}
