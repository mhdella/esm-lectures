% NB: use pdflatex to compile NOT pdftex.  Also make sure youngtab is
% there...

% converting eps graphics to pdf with ps2pdf generates way too much
% whitespace in the resulting pdf, so crop with pdfcrop
% cf. http://www.cora.nwra.com/~stockwel/rgspages/pdftips/pdftips.shtml




\documentclass[10pt,aspectratio=169,dvipsnames]{beamer}

\usetheme[color/block=transparent]{metropolis}

\usepackage[absolute,overlay]{textpos}
\usepackage{booktabs}

\usepackage{graphbox} %loads graphicx package

\usepackage{pbox}

\usepackage{adjustbox}

\usepackage[utf8]{inputenc}


\usepackage[scale=2]{ccicons}


\usepackage[official]{eurosym}


\usepackage{tikz}
\usetikzlibrary{arrows.meta}


\usepackage[europeanresistors,americaninductors]{circuitikz}



\setbeamerfont{alerted text}{series=\bfseries}
%\setbeamercolor{alerted text}{fg=BurntOrange}
%\setbeamercolor{alerted text}{fg=mDarkTeal}
%\setbeamercolor{alerted text}{fg=Brown}  % Goran
\setbeamercolor{alerted text}{fg=Mahogany}  % Goran

\setbeamercolor{background canvas}{bg=white}


\newcommand{\R}{\mathbb{R}}

\def\l{\lambda}
\def\m{\mu}
\def\d{\partial}
\def\cL{\mathcal{L}}
\def\co2{CO${}_2$}


\def\el{${}_{el}$}
\def\th{${}_{th}$}
\def\gas{${}_{gas}$}



\newif\ifgoat

\goattrue                       %



\newcommand*\rot{\rotatebox{90}}


\usepackage{pifont}
\newcommand*\OK{\ding{51}}


%use this to add space between rows
\newcommand{\ra}[1]{\renewcommand{\arraystretch}{#1}}

% for sources http://tex.stackexchange.com/questions/48473/best-way-to-give-sources-of-images-used-in-a-beamer-presentation

\setbeamercolor{framesource}{fg=gray}
\setbeamerfont{framesource}{size=\tiny}


\newcommand{\source}[1]{\begin{textblock*}{5cm}(10.5cm,8.35cm)
    \begin{beamercolorbox}[ht=0.5cm,right]{framesource}
        \usebeamerfont{framesource}\usebeamercolor[fg]{framesource} Source: {#1}
    \end{beamercolorbox}
\end{textblock*}}

\usepackage{hyperref}


%\usepackage[pdftex]{graphicx}


\graphicspath{{graphics/}}

\DeclareGraphicsExtensions{.pdf,.jpeg,.png,.jpg}



\def\goat#1{{\scriptsize\color{green}{[#1]}}}



\let\olditem\item
\renewcommand{\item}{%
\olditem\vspace{5pt}}

\title{Energy System Modelling\\ Summer Semester 2020, Lecture 12}
%\subtitle{---}
\author{
  {\bf Dr. Tom Brown}, \href{mailto:tom.brown@kit.edu}{tom.brown@kit.edu}, \url{https://nworbmot.org/}\\
  \emph{Karlsruhe Institute of Technology (KIT), Institute for Automation and Applied Informatics (IAI)}
}

\date{}


\titlegraphic{
  \vspace{0cm}
  \hspace{10cm}
    \includegraphics[trim=0 0cm 0 0cm,height=1.8cm,clip=true]{kit.png}

\vspace{5.1cm}

  {\footnotesize

  Unless otherwise stated, graphics and text are Copyright \copyright Tom Brown, 2020.
  Graphics and text for which no other attribution are given are licensed under a
  \href{https://creativecommons.org/licenses/by/4.0/}{Creative Commons
  Attribution 4.0 International Licence}. \ccby}
}

\begin{document}

\maketitle


\begin{frame}

  \frametitle{Table of Contents}
  \setbeamertemplate{section in toc}[sections numbered]
  \tableofcontents[hideallsubsections]
\end{frame}

\section{Sector Coupling: Invitation}


\begin{frame}
  \frametitle{What to do about variable renewables?}

  Backup energy costs money and may also cause CO${}_2$ emissions.

  Curtailing renewable energy is also a waste.

  We consider \alert{four options} to deal with variable renewables:
  \begin{enumerate}
  \item Smoothing stochastic variations of renewable feed-in over \alert{larger areas using networks}, e.g. the whole of European continent.
  \item Using \alert{storage} to shift energy from times of surplus to deficit.
  \item \alert{Shifting demand} to different times, when renewables are abundant.
    \item Consuming the electricity in \alert{other sectors}, e.g. transport or heating.
  \end{enumerate}

\alert{Optimisation} in energy networks is a tool to assess these options.

\end{frame}

\begin{frame}
  \frametitle{Sector coupling}

  In this lecture we will consider \alert{sector coupling}: the deeper
  coupling of electricity with other sectors, i.e. transport, heating
  and industry.

  In fact we will see that sector coupling is not just `an option for
  dealing with variable renewables' but is \alert{unavoidable} if we
  are going to reduce carbon dioxide emissions in the other
  sectors. It began decades ago with the coupling of power and
  heat in CHPs.

  Furthermore sector coupling involves both \alert{storage} (since in
  transport energy-dense fuels/batteries are required for vehicles; in
  heating some thermal and/or chemical storage may be unavoidable for cold snaps) and
  \alert{demand-side management} (e.g. for shifting battery electric
  vehicle charging, or shifting heat pump operation).


\end{frame}

\begin{frame}
  \frametitle{The Global Carbon Dioxide Challenge: Net-Zero Emissions by 2050}

  \begin{columns}[T]
\begin{column}{7.5cm}

  \includegraphics[width=8cm]{ipcc-sr15}


\end{column}
\begin{column}{6cm}
  \begin{itemize}
\item Scenarios for global CO$_2$ emissions that limit warming to 1.5$^\circ$C about industrial levels (\alert{Paris agreement})
\item Today emissions \alert{still rising}
\item Level of use of negative emission technologies (NET) depends on rate of progress
\item 2$^\circ$C target without NET also needs
  rapid fall by 2050
\item Common theme: \alert{net-zero by 2050}
\end{itemize}
\end{column}
  \end{columns}

    \source{\href{http://ipcc.ch/report/sr15/}{IPCC SR15 on 1.5C, 2018}}
\end{frame}

\begin{frame}
  \frametitle{The Greenhouse Gas Challenge: Net-Zero Emissions by 2050}

  Paris-compliant 1.5$^\circ$~C scenarios from European Commission - \alert{net-zero GHG in EU by 2050}

  \vspace{.4cm}

  %Figure 6 from https://ec.europa.eu/clima/sites/clima/files/docs/pages/com_2018_733_en.pdf
  \includegraphics[width=14cm]{eu-lts-net-zero.png}


    \source{\href{https://ec.europa.eu/clima/sites/clima/files/docs/pages/com_2018_733_en.pdf}{European Commission `Clean Planet for All', 2018}}
\end{frame}


\begin{frame}
  \frametitle{It's not just about electricity demand...}

  %GHG in 2016 are 4.291~Gt, CO2 is 3.489 ~Gt without LULUCF and without indirect
  %Global is 36.2 Gt (EU figure from carbonbrief excludes LULUCF, so assume global also)
%https://www.carbonbrief.org/analysis-global-co2-emissions-set-to-rise-2-percent-in-2017-following-three-year-plateau
  EU28 \co2{} emissions in 2016 (total 3.5~Gt \co2, 9.7\% of global):

  \centering
    % left bottom right top
  \includegraphics[trim=0 0.7cm 0 0.5cm,width=12cm]{EU28-emissions_pie-2016-CO2-190311.pdf}

%  ...\alert{but} wind and solar will dominate primary energy in all sectors, so electrification is critical.

  \source{Brown, data from \href{https://www.eea.europa.eu/data-and-maps/data/national-emissions-reported-to-the-unfccc-and-to-the-eu-greenhouse-gas-monitoring-mechanism-13}{EEA}}
\end{frame}



\begin{frame}
  \frametitle{...but electification of other sectors is critical for decarbonisation}

  Wind and solar dominate the expandable potentials for low-carbon
  energy provision, so \alert{electrification is essential} to decarbonise
  sectors such as transport and heating.

  \vspace{0.5cm}

  \begin{columns}[T]
\begin{column}{6cm}
  \includegraphics[trim=0 0cm 0 0cm,width=6cm,clip=true]{tesla.jpg}
\end{column}
\begin{column}{6cm}

  \includegraphics[trim=0 0cm 0 0cm,width=6cm,clip=true]{640px-Heat_Pump.jpg}
\end{column}
\end{columns}

    \vspace{0.7cm}
  Fortunately, these sectors can also offer crucial \alert{flexibility} back to the electricity system.


\source{Tesla; heat pump: \href{https://commons.wikimedia.org/w/index.php?curid=10795550}{Kristoferb at English Wikipedia}}

\end{frame}




\begin{frame}
  \frametitle{Daily variations: challenges and solutions}

  \begin{columns}[T]
    \begin{column}{5cm}
      \includegraphics[trim=0 0cm 0 0cm,width=5cm,clip=true]{DE-solar-day.pdf}
      \includegraphics[trim=0 0cm 0 0cm,width=5cm,clip=true]{DE-transport-day.pdf}
    \end{column}
    \begin{column}{4cm}
      Daily variations in supply and demand can be balanced by
      \begin{itemize}
      \item \alert{short-term storage} (e.g. batteries, pumped-hydro, small thermal storage)
      \item \alert{demand-side management} (e.g. battery electric vehicles,
        industry)
      \item \alert{east-west grids over multiple time zones}
      \end{itemize}

    \end{column}
    \begin{column}{5cm}
      \includegraphics[trim=0 0cm 0 0cm,width=5cm,clip=true]{pumped-storage-diagram-best1.jpg}

      \vspace{.5cm}

      \includegraphics[trim=0 0cm 0 0cm,width=5cm,clip=true]{tesla-charging.jpg}
    \end{column}
  \end{columns}

\end{frame}




\begin{frame}
  \frametitle{Synoptic variations: challenges and solutions}

  \begin{columns}[T]
    \begin{column}{5cm}
      \includegraphics[trim=0 0cm 0 0cm,width=5cm,clip=true]{2015-11-30-0300.png}

      \vspace{.2cm}

      \includegraphics[trim=0 0cm 0 0cm,width=5.5cm,clip=true]{DE-wind-month.pdf}
    \end{column}
    \begin{column}{4cm}
      Synoptic variations in supply and demand can be balanced by
      \begin{itemize}
      \item \alert{medium-term storage} (e.g. chemically with hydrogen or methane storage, thermal energy storage, hydro reservoirs)
      \item \alert{continent-wide grids}
      \end{itemize}

    \end{column}
    \begin{column}{5cm}
      \includegraphics[trim=0 0cm 0 0cm,width=4.5cm,clip=true]{1024px-Gasometer_in_East_London.jpg}

      \includegraphics[trim=0 0cm 0 0cm,width=4.5cm,clip=true]{europe_map.pdf}
    \end{column}
  \end{columns}

\end{frame}




\begin{frame}
  \frametitle{Seasonal variations: challenges and solutions}

  \begin{columns}[T]
    \begin{column}{5cm}
      \includegraphics[trim=0 0cm 0 0cm,width=5cm,clip=true]{DE-wind-solar-year.pdf}
      \includegraphics[trim=0 0cm 0 0cm,width=5cm,clip=true]{DE-heat-year.pdf}
    \end{column}
    \begin{column}{4cm}
      Seasonal variations in supply and demand can be balanced by
      \begin{itemize}
      \item \alert{long-term storage} (e.g. chemically with hydrogen or methane storage, long-term thermal energy storage, hydro reservoirs)
      \item \alert{north-south grids over multiple latitudes}
      \end{itemize}

    \end{column}
    \begin{column}{5cm}
      \includegraphics[trim=0 0cm 0 0cm,width=4.7cm,clip=true]{1024px-Gasometer_in_East_London.jpg}

      \vspace{.2cm}

      \includegraphics[trim=0 0cm 0 0cm,width=4.7cm,clip=true]{pit-zoom.png}
    \end{column}
  \end{columns}

\end{frame}


\section{Electricity, Heat in Buildings and Land Transport}


\begin{frame}
  \frametitle{Include other sectors: building heating and land transport}

  Electricity,  heating in buildings and land transport cover 77\% of 2015 CO$_2$ emissions:

  \centering
    % left bottom right top
  \includegraphics[trim=0 0.7cm 0 0.5cm,width=12cm]{EU28-emissions_pie-2016-CO2-190311.pdf}


  \source{Brown, data from \href{https://www.eea.europa.eu/data-and-maps/data/national-emissions-reported-to-the-unfccc-and-to-the-eu-greenhouse-gas-monitoring-mechanism-13}{EEA}}
\end{frame}





\begin{frame}{Efficiency of renewables and sector coupling}


  \includegraphics[width=\linewidth,trim=2.7cm 20cm 3.2cm 1.8cm,clip=true]{./graphics/bmwi-whitepaper-figure_18.pdf}

  \source{\href{https://www.bmwi.de/Redaktion/EN/Publikationen/whitepaper-electricity-market.html}{BMWi White Paper 2015}}
\end{frame}


\begin{frame}[fragile]
  \frametitle{Challenge: Heating and transport demand highly peaked}


  \begin{columns}[T]
\begin{column}{6cm}
  Compared to electricity, heating and transport are \alert{strongly peaked}.
  \begin{itemize}
  \item   Heating is strongly seasonal, but also with synoptic variations.
    \item   Transport has strong daily periodicity.
  \end{itemize}

  \vspace{.3cm}

  \includegraphics[trim=0 0cm 0 0cm,width=6cm,clip=true]{elec_demand.pdf}

\end{column}
\begin{column}{6cm}
  \includegraphics[trim=0 0cm 0 0cm,width=5.7cm,clip=true]{heat_demand.pdf}
  \includegraphics[trim=0 0cm 0 0cm,width=5.7cm,clip=true]{transport_profiles.pdf}
\end{column}
\end{columns}



\end{frame}

\begin{frame}[fragile]
  \frametitle{Sector Coupling}

  \alert{Idea}: Couple the electricity sector to heating and mobility.

  This enables decarbonisation of these sectors \alert{and} offers more flexibility to the power system.

  \vspace{0.2cm}

  \begin{columns}[T]
\begin{column}{6cm}
  \alert{Battery electric vehicles} can change their charging pattern to benefit the
  system and even feed back into the grid if necessary

  \includegraphics[trim=0 0cm 0 0cm,width=6cm,clip=true]{tesla.jpg}
\end{column}
\begin{column}{6cm}
  \alert{Heat} and \alert{synthetic fuels} are easier and cheaper to store than electricity, even over many months

  \vspace{0.5cm}

        \includegraphics[trim=0 0cm 0 0cm,width=6cm,clip=true]{pit-zoom}
\end{column}
\end{columns}


\end{frame}


\begin{frame}
  \frametitle{Power-to-Gas (P2G)}

  \begin{columns}[T]
    \begin{column}{4cm}
      \includegraphics[trim=0 0cm 0 0cm,width=5cm,clip=true]{pem_electrolyzer.png}


      \vspace{.4cm}

      \includegraphics[trim=0 0cm 0 0cm,width=5cm,clip=true]{Methanation_of_CO2_circle.png}
    \end{column}
    \begin{column}{7cm}

      Power-to-Gas/Liquid (P2G/L) describes concepts to use electricity to
      electrolyse water to \alert{hydrogen} H$_2$ (and oxygen O$_2$).
      We can combine hydrogen with carbon oxides to get
      \alert{hydrocarbons} such as methane CH$_4$ (main component of
      natural gas) or liquid fuels C$_n$H$_m$.

      Used for \alert{hard-to-defossilise sectors}:
            \begin{itemize}
            \item
              \alert{dense fuels} for transport (planes, ships)
            \item \alert{steel-making} \& \alert{chemicals industry}
              \item \alert{high-temperature heat} or \alert{heat for buildings}
                \item \alert{backup energy} for cold low-wind winter periods, i.e. as storage
            \end{itemize}
    \end{column}
  \end{columns}

\end{frame}



\begin{frame}
  \frametitle{Gas storage and networks}

  \begin{columns}[T]
    \begin{column}{5cm}
      \includegraphics[trim=0 0cm 0 0cm,width=4.5cm,clip=true]{salt_caverns.jpg}

      \vspace{.4cm}

      \includegraphics[trim=0 0cm 0 0cm,width=4.5cm,clip=true]{Gas-pipeline.jpg}
    \end{column}

    \begin{column}{7cm}
            \begin{itemize}
      \item  Gases and liquids are easy to \alert{store} and \alert{transport} than electricity.
        \item Storage capacity of the German natural gas network in terms of energy: ca 230 TWh. Europe wide it is 1100 TWh (see \href{https://agsi.gie.eu/}{\bf\color{blue}\underline{online table}}). In addition, losses in the gas network are small.
\item   (NB: Volumetric energy density of hydrogen, i.e. MWh/m$^3$, is around three times lower than natural gas.)
\item      Pipelines can carry many GW underground, out of sight.
            \end{itemize}
    \end{column}

\end{columns}

\end{frame}





\begin{frame}
  \frametitle{Sector coupling: A new source of flexibility}

  Couple the electricity sector (electric demand, generators,
  electricity storage, grid) to electrified
  transport and low-T heating demand in buildings (model covers 75\% of final energy consumption in 2014).
  Also allow production of synthetic hydrogen and methane.
% graphic generated in
\begin{figure}[!t]
\begin{adjustbox}{scale=0.70,trim=5 8.8cm 0 1cm}


  \begin{circuitikz}
  \draw (1.5,14.5) to [short,i^=electricity grid] (1.5,13);
  \draw [ultra thick] (-5,13) node[anchor=south]{electric bus} -- (6,13);
  \draw(2.5,13) |- +(0,0.5) to [short,i^=$$] +(2,0.5);
  \draw (0,-0.5) ;
  \draw (0.5,13) -- +(0,-0.5) node[sground]{};
  \draw (2.5,12) node[vsourcesinshape, rotate=270](V2){}
  (V2.left) -- +(0,0.6);
  \draw (2.5,11.2) node{generators};
    \node[draw,minimum width=1cm,minimum height=0.6cm,anchor=south west] at (3.4,11.9){storage};
    \draw (4,13) to (4,12.5);


  \draw [ultra thick] (-6,10) node[anchor=south]{transport} -- (-3,10);
  \draw (-5.5,10) -- +(0,-0.5) node[sground]{};
  \draw (-3.5,10) to [short,i_=${}$] (-3.5,13);
  \draw (-3.2,11.5)  node[rotate=90]{discharge};
  \draw (-4.5,13) to [short,i^=${}$] (-4.5,10);
  \draw (-4.2,11.5)  node[rotate=90]{charge};
  \node[draw,minimum width=1cm,minimum height=0.6cm,anchor=south west] at (-4.5,8.9){battery};
  \draw (-4,10) to (-4,9.5);

    \draw [ultra thick] (2,10) -- (6.5,10)  node[anchor=south]{heat};
  \draw (3.5,10) -- +(0,-0.5) node[sground]{};
  % esource (empty source) is a dipole, so remove the legs by making it connect a distance of its own width
  % follows: http://tex.stackexchange.com/questions/87275/use-circuitikz-voltage-source-icon-as-a-node
  \draw (4.5,9.35) to [esource] (4.5,8.5);
  \draw (4.5,10) -- (4.5,9.35);
  \draw (4.5,8.3) node{solar thermal};
  \draw (5,13) to [short,i^=heat pump;] (5,10);
  \draw (6.2,11) node{resistive heater};
  \node[draw,minimum width=1cm,minimum height=0.6cm,anchor=south west] at (5.5,8.9){hot water tank};
  \draw (6,10) to (6,9.5);


  \draw [ultra thick] (-2,10)  -- (0.5,10) node[anchor=south]{hydrogen};
  \draw (-1.5,13) to [short,i_=${}$] (-1.5,10);
    \draw (-1.2,11.5)  node[rotate=90]{electrolysis};
  \draw (-0.5,10) to [short,i^=${}$] (-0.5,13);
  \draw (-0.2,11.5)  node[rotate=90]{fuel cell};
  \draw (-1,10) to (-1,9.5);
  \node[draw,minimum width=1cm,minimum height=0.6cm,anchor=south west] at (-1.5,8.9){storage};

  \draw (0,10) to [short,i_=${}$] (0,8);
  \draw [ultra thick] (-1,8) node[anchor=south]{methane} -- (3,8);
  \draw(2,8) to (2,7.3) to [short,i^=gas grid] (4,7.3);
  \draw (1.5,8) to [short,i_=${}$] (1.5,13);
  \draw (2.5,8) to [short,i_=${}$] (2.5,10);
  \node[draw,minimum width=1cm,minimum height=0.6cm,anchor=south west] at (0.5,6.9){storage};
  \draw (1,8) to (1,7.5);
  \draw (0.3,9)  node[rotate=90]{methanation};
  \draw (1.8,9.2)  node[rotate=90]{generator/CHP};
  \draw (2.8,9)  node[rotate=90]{boiler/CHP};
  \end{circuitikz}

\end{adjustbox}
\end{figure}

\end{frame}



\begin{frame}
  \frametitle{Modelling: extend network graph for energy conversion processes}

  Extend the network graph with nodes $i$ for each energy carrier (hydrogen, methane, low-temperature heat, etc.). The nodes represent sites of energy conservation.

  Edges $\ell$ now represent energy conversion between energy carriers (such as heat pumps, electrolysers, fuel cells or gas boilers).

  They are represented like lines but with an efficiency $\eta_{\ell,t}$ that modifies the incidence matrix:
\begin{equation*}
      \sum_s g_{i,s,t} - \sum_\ell \alpha_{i\ell t}f_{\ell,t} = d_{i,t}  \hspace{1cm}\leftrightarrow\hspace{1cm} \l_{i,t}
\end{equation*}
Now $\alpha_{i\ell t} = 1$ if $\ell$ starts at $i$, $\alpha_{i\ell t} = -\eta_{\ell,t}$ if $\ell$ ends at $i$ and zero otherwise.

Note that $\eta_{\ell,t}$ can be time-dependent for processes that change their efficiency over time, like heat pumps which change with the outside temperature.

They are usually defined to be uni-directional:
\begin{equation*}
  0 \leq f_{\ell,t} \leq F_{\ell}
\end{equation*}
[In PyPSA energy conversion is represented with Link objects.]


\end{frame}



\begin{frame}
  \frametitle{Transport sector: Electrification of Transport}
\begin{columns}[T]
\begin{column}{7cm}
    \includegraphics[trim=0 0cm 0 0cm,width=8cm,clip=true]{transport_profiles.pdf}

    Weekly profile for the transport demand based on statistics gathered by the German Federal Highway Research Institute (BASt).

\end{column}
\begin{column}{7cm}
  \begin{itemize}
  \item All road and rail transport in each country is electrified, where it is not already electrified
    \item Because of higher efficiency of electric motors, final
      energy consumption 3.5 times lower than today at 1102~TWh\el/a for the 30
      countries
            \item In model can replace Electric Vehicles (EVs) with Fuel Cell Vehicles (FCVs) consuming hydrogen. Advantage: hydrogen cheap to store. Disadvantage: efficiency of fuel cell only 60\%, compared to 90\% for battery discharging.
  \end{itemize}
\end{column}
\end{columns}

\end{frame}



\begin{frame}
  \frametitle{Transport sector: Battery Electric Vehicles}
\begin{columns}[T]
\begin{column}{7cm}
    \includegraphics[trim=0 0cm 0 0cm,width=8cm,clip=true]{bev_availability.pdf}

    Availability (i.e. fraction of vehicles plugged in) of Battery Electric Vehicles (BEV).
\end{column}
\begin{column}{7cm}
  \begin{itemize}
  \item Passenger cars to Battery Electric Vehicles (BEVs), 50~kWh battery available and 11~kW charging power
  \item Can participate in DSM and V2G, depending on scenario (state of charge returns to at least 75\% every morning)
    \item All BEVs have time-dependent availability, averaging 80\%, max 95\% (at night)
    \item No changes in consumer behaviour assumed (e.g. car-sharing/pooling)
    \item BEVs are treated as exogenous (capital costs NOT included in calculation)
  \end{itemize}
\end{column}
\end{columns}

\end{frame}


\begin{frame}
  \frametitle{Coupling Transport to Electricity in European Model with 95\% Less CO$_2$}
  \begin{columns}[T]
    \begin{column}{7cm}
      \includegraphics[width=8.3cm]{RGI-t-0.pdf}


    \end{column}
    \begin{column}{7cm}
      \begin{itemize}
      \item Include transport demand in 30-node PyPSA electricity model for Europe
      \item Apply 95\% CO$_2$ reduction vs 1990 to both electricity and transport
      \item If all road and rail transport is electrified, electrical demand increases 37\%
      \item Costs increase 41\% because charging profiles are very peaked (NB: distribution grid costs NOT included)
      \item Stronger preference for PV and storage in system mix because of daytime peak
      \item Can now use flexible charging
      \end{itemize}
    \end{column}
  \end{columns}
\end{frame}




\begin{frame}
  \frametitle{Using Battery Electric Vehicle Flexibility}
  \begin{columns}[T]
    \begin{column}{9cm}
      \includegraphics[width=9cm]{RGI-transport-0.pdf}


    \end{column}
    \begin{column}{6cm}
      \begin{itemize}
      \item Shifting the charging time can reduce system costs by up to 14\%.
      \item If only 25\% of vehicles participate: already a 10\% benefit.
      \item Allowing battery EVs to feed back into the grid (V2G) reduces costs by a further 10\%.
      \item This removes case for stationary batteries and allows more solar.
      \item If fuel cells replace electric vehicles, hydrogen electrolysis increases costs because of conversion losses.
      \end{itemize}
    \end{column}
  \end{columns}
\end{frame}












\begin{frame}
  \frametitle{Heating sector: Many Options with Thermal Energy Storage (TES)}
\begin{columns}[T]
\begin{column}{7cm}

    \includegraphics[trim=0 0cm 0 0cm,width=8cm,clip=true]{heat_demand.pdf}

    Heat demand profile from 2011 in all 30 countries using population-weighted average daily T in each country, degree-day approx. and scaled to Eurostat total heating demand.


\end{column}
\begin{column}{7cm}
  \begin{itemize}
  \item All space and water heating in the residential and services
    sectors is considered, with no additional efficiency measures
    (conservative) - total heating demand is  3585~TWh\th/a.
  \item Heating demand can be met by heat pumps, resistive heaters, gas boilers, solar thermal, Combined-Heat-and-Power (CHP) units. No industrial waste heat.
    \item Thermal Energy Storage (TES) is available to the system as hot water tanks.
  \end{itemize}
\end{column}
\end{columns}

\end{frame}






\begin{frame}
  \frametitle{Centralised District Heating versus Decentralised Heating for Buildings}

  We model both fully decentralised heating and cases where up to 45\%
  of heat demand is met with district heating in northern countries.
  Heating technology options for buildings:

\begin{columns}[T]
\begin{column}{5cm}
  \alert{Decentral individual heating} can be supplied by:
  \begin{itemize}
    \item Air- or Ground-sourced heat pumps
  \item Resistive heaters
  \item Gas boilers
    \item Small solar thermal
  \item Water tanks with short time constant $\tau = 3$ days
  \end{itemize}
\end{column}
\begin{column}{5cm}
  \alert{Central heating} can be supplied via district heating networks by:
  \begin{itemize}
    \item Air-sourced heat pumps
\item Resistive heaters
\item Gas boilers
        \item Large solar thermal
  \item Water tanks with long time constant $\tau = 180$ days
\item CHPs
  \end{itemize}
\end{column}

\begin{column}{5cm}
  CHP feasible dispatch:
    \includegraphics[trim=0 0cm 0 0cm,width=5cm,clip=true]{chp_feasible.pdf}
\end{column}

\end{columns}
  Building renovations can be co-optimised to reduce space heating demand.

\end{frame}



\begin{frame}
  \frametitle{Heat pumps}

  \alert{Heat pumps} use external work (usually electricity) to move
  thermal energy in the opposite direction of spontaneous heat
  transfer, e.g. by absorbing heat from a cold space (\alert{source}) and release it into a
  warmer one (\alert{sink}).

  When the sink is a building, the source is usually the outside air or ground.

\begin{columns}[T]
\begin{column}{6cm}
  Air-source heat pumps (ASHP):

  \vspace{.2cm}

  \includegraphics[trim=0 0cm 0 0cm,width=6.5cm,clip=true]{ashp.png}

\end{column}
\begin{column}{6cm}
  Ground-source heat pumps (GSHP):

  \vspace{.2cm}

  \includegraphics[trim=0 0cm 0 0cm,width=6.5cm,clip=true]{gshp.png}

\end{column}
\end{columns}

\source{\href{http://dx.doi.org/10.1039/c2ee22653g}{Staffell et al, 2012}}

\end{frame}

\begin{frame}
  \frametitle{Heat pumps}

  The \alert{coefficient of performance} (COP) is defined as the ratio:
  \begin{equation*}
    COP = \frac{\textrm{thermal energy moved from source to sink}}{\textrm{input work (electricity)}} \propto \frac{1}{T_{\textrm{sink}} - T_{\textrm{source}}}
  \end{equation*}


  \centering
  \includegraphics[trim=0 0cm 0 0cm,width=12cm,clip=true]{cop-staffel.png}


\source{\href{http://dx.doi.org/10.1039/c2ee22653g}{Staffell et al, 2012}}
\end{frame}

\begin{frame}
  \frametitle{Heat pumps}

  Example of time-dependent COP for air-source and ground-source heat pumps in a location in Germany. The ground temperature is more stable over the year, leading to a stable COP.

  \centering
  \includegraphics[trim=0 0cm 0 0cm,width=14cm,clip=true]{de-cop.pdf}
\end{frame}


\begin{frame}
  \frametitle{Long-duration thermal energy storage}

  In Vojens, Denmark, an enormous pit storage of 203,000~m$^3$ is charged in summer with hot water at 80-95~C using 70,000~m$^2$ of solar thermal collectors, to provide heat to the district heating network in winter.

  \vspace{.7cm}
    \begin{columns}[T]
\begin{column}{6cm}


        \includegraphics[trim=0 0cm 0 0cm,width=6cm,clip=true]{pit-zoom}
\end{column}
\begin{column}{6cm}

        \includegraphics[trim=0 0cm 0 0cm,width=6cm,clip=true]{vojens_seasonal_storage_august_2014.jpg}
\end{column}
    \end{columns}

    \source{\href{https://www.solarthermalworld.org/news/denmark-37-mw-field-203000-m3-storage-underway}{Solar Thermal World}}
\end{frame}


\begin{frame}
  \frametitle{Cost and other assumptions}

  \begin{table}
\centering
\begin{tabular}{@{}lrlrrr@{}}
\toprule
Quantity                & O'night cost [\euro]  &Unit & FOM [\%/a] & Lifetime [a] & Efficiency \\
\midrule
GS Heat pump decentral & 1400 & kW\th  & 3.5& 20 \\
AS Heat pump decentral & 1050 & kW\th  & 3.5& 20 \\
AS Heat pump central & 700 & kW\th  & 3.5& 20 \\
Resistive heater & 100  & kW\th  & 2& 20 & 0.9\\
Gas boiler decentral & 175  & kW\th  & 2& 20 & 0.9 \\
Gas boiler central & 63  & kW\th  & 1& 22 & 0.9 \\
CHP & 650 & kW\el & 3& 25\\
Central water tanks & 30 & m${}^3$  & 1& 40 & $\tau = 180$d\\
District heating & 220 & kW\th & 1 & 40 &  \\
Methanation+DAC & 1000 & kW$_{H_2}$ & 3 & 25  & 0.6\\
%Gas plant efficiency    &30     &\%     \\
%Interest rate           &7      &\%     \\
\bottomrule
\end{tabular}
\end{table}
  Costs oriented towards Henning \& Palzer (2014, Fraunhofer ISE) and Danish Energy Database
\end{frame}


\begin{frame}
  \frametitle{Coupling Heating to Transport and Electricity: Electricity Demand}
  \begin{columns}[T]
    \begin{column}{8cm}
      \includegraphics[width=9cm]{RGI-supply_energy.pdf}

    \end{column}
    \begin{column}{6cm}
      \begin{itemize}
      \item To 4062~TWh\el/a demand from electricity and transport, add 3585~TWh\th/a heating demand
      \item With 95\% CO$_2$ reduction, much of the heating demand is met via electricity, but with high efficiency from heat pumps
      \item Electricity demand 80\% higher than current electricity demand
        %5600/3100
      \item Energy savings from building retrofitting can reduce this total
      \end{itemize}
    \end{column}
  \end{columns}
\end{frame}



\begin{frame}
  \frametitle{Coupling Heating to Transport and Electricity: Costs}
  \begin{columns}[T]
    \begin{column}{9cm}
      \includegraphics[width=9cm]{RGI-h-0.pdf}

    \end{column}
    \begin{column}{5cm}
      \begin{itemize}
      \item Costs jump by 117\% to cover new energy supply and heating infrastructure
      \item 95\% \co2{} reduction means most heat is generated by heat pumps using renewable electricity
      \item Cold winter weeks with high demand, low wind, low solar and low heat pump COP mean backup gas boilers required
      \end{itemize}
    \end{column}
  \end{columns}
\end{frame}


\begin{frame}
  \frametitle{Cold week in winter}

    \begin{columns}[T]
    \begin{column}{6cm}

  \centering
\includegraphics[trim=0 0cm 0 0cm,width=5.5cm,clip=true]{cold_week-base-0-DE}

\includegraphics[trim=0 0cm 0 0cm,width=5.5cm,clip=true]{cold_week-base-0_urban_heat-DE}
    \end{column}
    \begin{column}{7cm}
      \vspace{1.5cm}
        There are difficult periods in winter with:
        \begin{itemize}
        \item \alert{Low} wind and solar generation
        \item \alert{High} space heating demand
        \item \alert{Low} air temperatures, which are bad for air-sourced heat pump performance
        \end{itemize}

        Solution: \alert{backup gas boilers} burning either natural gas, or
        synthetic methane.
    \end{column}
    \end{columns}
\end{frame}



\begin{frame}
  \frametitle{Using heating flexibility}
  \begin{columns}[T]
    \begin{column}{8.3cm}
      \includegraphics[width=9cm]{RGI-heating-0.pdf}
    \end{column}
    \begin{column}{6cm}

      \vspace{.5cm}
        Successively activating couplings and flexibility \alert{reduces costs} by 28\%.
        These options include:
        \begin{itemize}
          \item  production of \alert{synthetic methane}
          \item  centralised \alert{district heating} in areas with dense heat demand
          \item long-term \alert{thermal energy storage} (TES)  in district heating networks
          \item \alert{demand-side management} and vehicle-to-grid from battery electric vehicles (BEV)
        \end{itemize}
    \end{column}
  \end{columns}
\end{frame}



\begin{frame}
  \frametitle{Cold week in winter: inflexible (left); smart (right)}

  \begin{columns}[T]
    \begin{column}{6.5cm}
  \centering
\includegraphics[trim=0 0cm 0 0cm,width=5.5cm,clip=true]{cold_week-base-0-DE}

\includegraphics[trim=0 0cm 0 0cm,width=5.5cm,clip=true]{cold_week-base-0_urban_heat-DE}
    \end{column}
    \begin{column}{6.5cm}
  \centering

  \includegraphics[trim=0 0cm 0 0cm,width=5.5cm,clip=true]{cold_week-central-tes-0-DE}

  \includegraphics[trim=0 0cm 0 0cm,width=5.5cm,clip=true]{cold_week-central-tes-0_urban_heat-DE}
    \end{column}
  \end{columns}
\end{frame}





\begin{frame}
  \frametitle{Sector Coupling with All Extra Flexibility (V2G and TES)}

  Benefit of cross-border transmission is weaker with full sector flexibility (right) than with inflexible sector coupling (left); comes close to today's costs of around \euro~377~billion per year

  \centering
  \includegraphics[width=13cm]{transmission.pdf}
\end{frame}



\begin{frame}
  \frametitle{Spatial distribution of primary energy for All-Flex-Central}

  Including optimal transmission sees a shift of energy production to wind in Northern Europe.

\begin{columns}[T]
  \begin{column}{7cm}

    \vspace{0.5cm}
  \includegraphics[width=7.6cm]{spatial-all_flex-central-0.pdf}

  \end{column}
  \begin{column}{7cm}

    \vspace{0.5cm}
  \includegraphics[width=7.6cm]{spatial-all_flex-central-opt.pdf}

  \end{column}
\end{columns}
\end{frame}


\begin{frame}
  \frametitle{Storage energy levels: different time scales}

\begin{columns}[T]
  \begin{column}{9cm}

    \vspace{0.5cm}
  \includegraphics[width=9cm]{scales.pdf}

  \end{column}

  \begin{column}{6cm}
    \begin{itemize}
    \item Methane storage is depleted in winter, then replenished
      throughout the summer with synthetic methane
    \item Hydrogen storage fluctuates every 2–3 weeks, dictated by
      wind variations
    \item Long-Term Thermal Energy Storage (LTES) has a dominant
      seasonal pattern, with synoptic-scale fluctuations are
      super-imposed
    \item Battery Electric Vehicles (BEV) and battery storage
      vary daily
    \end{itemize}

  \end{column}

\end{columns}
\end{frame}





\begin{frame}{Pathway down to zero emissions in electricity, heating and transport}
\begin{columns}[T]
  \begin{column}{9cm}

    \includegraphics[width=9cm]{RGI-CO2-twin}

  \end{column}


  \begin{column}{6cm}

    \vspace{0.5cm}

    If we look at investments to eradicate \co2{} emissions in
    electricity, heating and transport we see:

    \begin{itemize}
    \item Electricity and transport are decarbonised first
    \item Transmission increasingly important below 30\%
    \item Heating comes next with expansion of heat pumps below 20\%
    \item Below 10\%, power-to-gas solutions replace natural gas
    \end{itemize}

  \end{column}

\end{columns}

\end{frame}


\begin{frame}{CO$_2$ price rises to displace cheap natural gas}

  \centering
    \includegraphics[width=11.245cm]{co2_price.pdf}

\end{frame}


\begin{frame}{Electricity price statistics: zero-price hours gone thanks to P2G}

  \centering
    \includegraphics[width=11.245cm]{price_statistics.pdf}

\end{frame}



\begin{frame}{Curtailment also much reduced}

  \centering
    \includegraphics[width=11.245cm]{curtailment.pdf}

\end{frame}



\begin{frame}{Market values relative to average load-weighted price re-converge}

  \centering
    \includegraphics[width=11.245cm]{market_values.pdf}

\end{frame}




\begin{frame}{Gas production/consumption tightly coupled to price}

  \centering
    \includegraphics[width=11cm]{gas_v_prices.pdf}

\end{frame}


\begin{frame}
  \frametitle{More Details in Papers}

  For more details, see the following papers:
  \begin{itemize}
  \item   Synergies of sector coupling and transmission reinforcement in a cost-optimised, highly renewable European energy system, \href{https://arxiv.org/abs/1801.05290}{\bf\color{blue}\underline{link}} (2018).
    \item Sectoral Interactions as Carbon Dioxide Emissions Approach Zero in a Highly-Renewable European Energy System, \href{https://doi.org/10.3390/en12061032}{\bf\color{blue}\underline{link}} (2019).
  \end{itemize}


\end{frame}

\section{Industry, Shipping and Aviation}


\begin{frame}
  \frametitle{CO$_2$ direct emissions from industry by sector in Europe}


\begin{columns}[T]
  \begin{column}{9.5cm}

    \includegraphics[width=10cm]{CO2_emissions_sectors.png}

  \end{column}


  \begin{column}{4cm}

    \vspace{0.2cm}

    \begin{itemize}
    \item Non-ferrous metals: mainly aluminium, but also copper, lead etc.
    \item Non-metallic minerals: mainly cement, ceramics and glass
    \item Emissions come from combustion of fossil fuels for heat, as
      well as \alert{process emissions} from chemical reactions
    \end{itemize}

  \end{column}

\end{columns}

\end{frame}


\begin{frame}
  \frametitle{Process heat demand in Europe by sector and temperature}

  Temperatures $>$100~C not accessible via regular heat pumps. Need direct electrification, biomass, synthetic fuels (hydrogen, methane), nuclear or carbon capture.

    \includegraphics[width=14cm]{process_heat_demand.png}


\source{\href{https://link.springer.com/article/10.1007\%2Fs12053-017-9571-y}{Rehfeldt et al, 2017}}
\end{frame}

\begin{frame}
  \frametitle{Iron and steel: direct reduce with hydrogen instead of coke}
\begin{columns}[T]
  \begin{column}{8cm}

    \includegraphics[width=9cm]{hybrit.jpg}

  \end{column}


  \begin{column}{5cm}

    \vspace{0.1cm}

    \begin{itemize}
    \item Normally coke is used as a reducing agent in blast furnaces for smelting iron ore
      \begin{equation*}
        2\textrm{Fe}_2\textrm{O}_3 + 3\textrm{C} \to 4\textrm{Fe} + 3\textrm{CO}_2
      \end{equation*}
    \item Instead: use hydrogen as the reducing agent
      \begin{equation*}
        \textrm{Fe}_2\textrm{O}_3 + 3\textrm{H}_2 \to 2\textrm{Fe} + 3\textrm{H}_2\textrm{O}
      \end{equation*}
    \item Should scale up in late 2020s and 2030s.
    \item See \href{https://doi.org/10.1016/J.JCLEPRO.2018.08.279}{\bf\color{blue}\underline{Vogl et al, 2018}}.
    \end{itemize}

  \end{column}

\end{columns}
\source{\href{http://www.hybritdevelopment.com/steel-making-today-and-tomorrow}{HYBRIT project}}
\end{frame}

\begin{frame}
  \frametitle{Cement}

  Cement is used in construction to make concrete. CO$_2$ is emitted from fossil fuels to provide process heat and from the calcination reaction for fossil limestone
  \begin{equation*}
        \textrm{CaCO}_3  \to \textrm{CaO} + \textrm{CO}_2
  \end{equation*}
  This is the biggest source of \alert{process emissions} in industry in Europe. While we can replace process heat with low-carbon sources, process emissions are harder. Unless alternatives can be found for cement, this CO$_2$ can be captured and either sequestered underground (CCS) or used (CCU) to make chemicals like methane, liquid hydrocarbons or methanol.

  \centering
  \includegraphics[width=13cm]{CemKilnKiln.jpg}

  \source{Wikipedia}
\end{frame}




\begin{frame}
  \frametitle{Chemicals}
\begin{columns}[T]
  \begin{column}{8cm}

    \includegraphics[width=9cm]{chemicals.png}

  \end{column}


  \begin{column}{6cm}

    \begin{itemize}
    \item Fossil fuels are used for process heat in the chemicals industry, but also as a \alert{feedstock} for chemicals like ammonia (NH$_3$), ethylene (C$_2$H$_4$) and methanol (CH$_3$OH or MeOH)
    \item Ammonia, used for fertiliser, can be made from hydrogen and nitrogen using the Haber-Bosch process
    \item Ethylene, used for plastics, can be made by steam cracking from naphtha or ethane
    \end{itemize}

  \end{column}

\end{columns}
\source{\href{https://dechema.de/dechema_media/Downloads/Positionspapiere/Technology_study_Low_carbon_energy_and_feedstock_for_the_European_chemical_industry-p-20002750.pdf}{DECHEMA}}
\end{frame}



\begin{frame}
  \frametitle{Power to Transport Fuels}

    \begin{columns}[T]
    \begin{column}{7cm}
      \includegraphics[width=7cm]{fuel_density-science.png}
    \end{column}
    \begin{column}{7cm}
      \begin{itemize}
      \item  Hydrogen has a very good gravimetric density (MJ/kg) but poor volumetric density (MJ/L).
      \item      Liquid hydrocarbons provide much better volumetric density for e.g. aviation.
      \item    WARNING: This graphic shows the thermal content of the fuel, but the \alert{conversion efficiency} of e.g. an electric motor for battery electric or fuel cell vehicle is much better than an internal combustion engine.
      \end{itemize}
      \end{column}
    \end{columns}
    \source{\href{https://doi.org/10.1126/science.aas9793}{Davis et al, 2018}}
\end{frame}


\begin{frame}
  \frametitle{Defossilising non-electric industry processes, aviation, shipping}

  We assume higher recycling levels as well as process- and fuel-switching (under review):
    \begin{table}
\centering
\begin{tabular}{@{}lp{10cm}@{}}
  \toprule
  %Sector & Assumption  \\
  %\midrule
  Iron \& Steel & 70\% from scrap, rest from direct reduction with 1.7~MWhH$_2$/tSteel $+$ electric arc (process emissions 0.03~tCO$_2$/tSteel)\\
  Aluminium & 80\% recycling, for rest: methane for high-enthalpy heat (bauxite to alumina) followed by electrolysis (process emissions 1.5 tCO$_2$/tAl) \\
  Cement &  Waste and solid biomass for heat; capture process emissions \\
  Ceramics \& other NMM &  Electrification\\
  Chemicals & Synthetic methane, synthetic naphtha and hydrogen \\
  Other industry & Electrification; process heat from biomass\\
  Shipping & Liquid hydrogen  (could be replaced by other liquid fuels) \\
  Aviation & Kerosene from Fischer-Tropsch  \\
  \bottomrule
\end{tabular}
\end{table}

 Carbon is tracked through system: 90\% of industrial emissions are captured; direct air capture (DAC); synthetic methane and liquid hydrocarbons; transport and sequestration 20~\euro/tCO$_2$

\end{frame}


\begin{frame}
  \frametitle{Including all sectors needs careful management of carbon}

  \centering
  \includegraphics[width=8.5cm]{20200223_multisector_figure.pdf}

\end{frame}

\begin{frame}
  \frametitle{Worldwide trade in synthetic fuels}

  Today fossil fuels are traded across the globe. Electrolytic-hydrogen-based synthetic fuels (e.g. hydrogen, ammonia, methane, liquid hydrocarbons and methanol) could also be piped/shipped worldwide.  Possible future scenario for hydrogen trade from Helmholtz colleagues at FZJ IEK-3:

  \centering
  \includegraphics[width=12cm]{h2_worldwide.png}


\source{\href{https://www.preprints.org/manuscript/202002.0100/v1}{Heuser et al, 2020}}
\end{frame}

\section{Open Energy Modelling}


\begin{frame}
  \frametitle{What is open modelling?}

  \alert{Open energy modelling} means modelling with open software, open data and open publishing.

  \alert{Open} means that anybody is free to download the software/data/publications, inspect it, machine process it, share it with others, modify it, and redistribute the changes.

  This is typically done by uploading the model to an online platform with an \alert{open licence} telling users what their reuse rights are.

  The \alert{whole pipeline} should be open:
  \vspace{0.5cm}

  \centering
  \includegraphics[width=13cm]{openmod}

  \vspace{0.1cm}
\end{frame}


\begin{frame}
  \frametitle{Why open modelling?}

  Openness \dots
  \begin{itemize}
  \item increases \alert{transparency}, \alert{reproducibility}
    and \alert{credibility}, which lead to better research and policy
    advice  (no more `black boxes' determining hundreds of billions of energy spending)
  \item  reduces
    \alert{duplication of effort} and frees time/money  to develop
    \alert{new ideas}
  \item \emph{can} improve research \alert{quality} through feedback and correction
  \item allows easier \alert{collaboration} (no need for contracts, NDAs, etc.)
  \item is a \alert{moral imperative} given that much of the work is publicly funded
  \item puts pressure on \alert{official data holders} to open up
  \item is essential given the increasing \alert{complexity} of the energy system - we all need data from different domains (grids, buildings, transport, industry) and cannot collect it alone
  \item can increase \alert{public acceptance} of difficult infrastructure trade-offs
  \end{itemize}
  {\tiny See also  S.~Pfenninger et al, `\href{https://doi.org/10.1016/j.enpol.2016.11.046}{The importance of open data and software: Is energy research lagging behind?},' Energy Policy, V101, p211, 2017 and S.~Pfenninger, `\href{https://dx.doi.org/10.1038/542393a}{Energy scientists must show their workings},' Nature, V542, p393, 2017.}

 % Reagan used the Russian proverb: ``trust, but verify''

\end{frame}




\begin{frame}
  \frametitle{openmod: overview}

  There's an initiative for that! Sign up for the mailing list / come to the next workshop:
  \vspace{0.1cm}

  \centering
  \includegraphics[width=12cm]{openmod-logo-large.png}

  \large
  \textbf{\href{http://openmod-initiative.org/}{openmod-initiative.org}}

  \normalsize

\begin{itemize}
  \item \alert{grass roots community} of open energy modellers from universities,
  research institutions and the interested public
  \item 700+ participants from all continents except Antarctica
  \item first meeting Berlin 18–19 September 2014
  \item promoting \alert{open code}, \alert{open data} and \alert{open science} in energy modelling
\end{itemize}

  \source{openmod initiative}

\end{frame}




\begin{frame}
  \frametitle{Python for Power System Analysis (PyPSA)}

  Our free software PyPSA is available online at \url{https://pypsa.org/} and on
  github. It can do:

\begin{columns}[T]
  \begin{column}{6cm}

  \begin{itemize}
  \item Static \alert{power flow}
  \item \alert{Linear optimal power flow} (LOPF) (multiple periods, unit commitment,
    storage, coupling to other sectors)
    \item \alert{Security-constrained LOPF}
    \item Total energy system \alert{capacity expansion optimisation}
  \end{itemize}

  It has models for storage, meshed AC grids, meshed DC grids, hydro plants, variable renewables and sector coupling.

  \end{column}

\begin{column}{7cm}

  \vspace{0.3cm}
\centering
\includegraphics[width=8cm]{lmp}


\end{column}
\end{columns}

\end{frame}


\begin{frame}
  \frametitle{Python for Power System Analysis: Worldwide Usage}

  PyPSA is used worldwide by \alert{dozens of research institutes and companies} (TU Delft, Shell, Fraunhofer ISE, DLR Oldenburg, FZJ, TU Berlin, RLI, TransnetBW, TERI, Flensburg Uni, Saudi Aramco, Edison Energy, spire and many others). Visitors to the website:

  \centering
  \includegraphics[width=12cm]{pypsa-170322}


\end{frame}



\section{Conclusions}

\begin{frame}{Conclusions}

  \begin{itemize}
    \item Meeting \alert{Paris targets} is much more urgent than widely recognised

    \item There are \alert{lots of cost-effective solutions} thanks to falling price of renewables

    \item \alert{Electrification of other energy sectors} like heating,
      transport and industry is important (direct or indirect with synthetic fuels),  to take advantage of
      low-carbon electricity

    \item \alert{Grid helps} to make CO2 reduction easier $=$ cheaper


    \item \alert{Cross-sectoral} approaches are important to reduce
      CO2 emissions \alert{and} for flexibility

    \item \alert{Policy prerequisites}: high, increasing and
      transparent \alert{price for \co2{} pollution}; financial and regulatory support for \alert{new technologies} (heat pumps, hydrogen for steel)

    \item The energy system is complex and contains some uncertainty
      (e.g. cost developments, scaleability of power-to-gas, consumer behaviour), so
      \alert{openness is critical}

  \end{itemize}

\end{frame}

\end{document}
